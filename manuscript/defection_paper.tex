\documentclass[12pt,]{article}
\usepackage[]{tgpagella}
\usepackage{amssymb,amsmath}
\usepackage{ifxetex,ifluatex}
\usepackage{fixltx2e} % provides \textsubscript
\ifnum 0\ifxetex 1\fi\ifluatex 1\fi=0 % if pdftex
  \usepackage[T1]{fontenc}
  \usepackage[utf8]{inputenc}
\else % if luatex or xelatex
  \ifxetex
    \usepackage{mathspec}
  \else
    \usepackage{fontspec}
  \fi
  \defaultfontfeatures{Ligatures=TeX,Scale=MatchLowercase}
\fi
% use upquote if available, for straight quotes in verbatim environments
\IfFileExists{upquote.sty}{\usepackage{upquote}}{}
% use microtype if available
\IfFileExists{microtype.sty}{%
\usepackage{microtype}
\UseMicrotypeSet[protrusion]{basicmath} % disable protrusion for tt fonts
}{}
\usepackage[margin=1in]{geometry}
\usepackage{hyperref}
\hypersetup{unicode=true,
            pdftitle={I Didn't Sign Up for This: Repression and the Fragmentation of Regime Forces},
            pdfborder={0 0 0},
            breaklinks=true}
\urlstyle{same}  % don't use monospace font for urls
\usepackage{longtable,booktabs}
\usepackage{graphicx,grffile}
\makeatletter
\def\maxwidth{\ifdim\Gin@nat@width>\linewidth\linewidth\else\Gin@nat@width\fi}
\def\maxheight{\ifdim\Gin@nat@height>\textheight\textheight\else\Gin@nat@height\fi}
\makeatother
% Scale images if necessary, so that they will not overflow the page
% margins by default, and it is still possible to overwrite the defaults
% using explicit options in \includegraphics[width, height, ...]{}
\setkeys{Gin}{width=\maxwidth,height=\maxheight,keepaspectratio}
\IfFileExists{parskip.sty}{%
\usepackage{parskip}
}{% else
\setlength{\parindent}{0pt}
\setlength{\parskip}{6pt plus 2pt minus 1pt}
}
\setlength{\emergencystretch}{3em}  % prevent overfull lines
\providecommand{\tightlist}{%
  \setlength{\itemsep}{0pt}\setlength{\parskip}{0pt}}
\setcounter{secnumdepth}{0}
% Redefines (sub)paragraphs to behave more like sections
\ifx\paragraph\undefined\else
\let\oldparagraph\paragraph
\renewcommand{\paragraph}[1]{\oldparagraph{#1}\mbox{}}
\fi
\ifx\subparagraph\undefined\else
\let\oldsubparagraph\subparagraph
\renewcommand{\subparagraph}[1]{\oldsubparagraph{#1}\mbox{}}
\fi

%%% Use protect on footnotes to avoid problems with footnotes in titles
\let\rmarkdownfootnote\footnote%
\def\footnote{\protect\rmarkdownfootnote}

%%% Change title format to be more compact
\usepackage{titling}

% Create subtitle command for use in maketitle
\newcommand{\subtitle}[1]{
  \posttitle{
    \begin{center}\large#1\end{center}
    }
}

\setlength{\droptitle}{-2em}

  \title{I Didn't Sign Up for This: Repression and the Fragmentation of Regime
Forces}
    \pretitle{\vspace{\droptitle}\centering\huge}
  \posttitle{\par}
    \author{David Bowden\\
University of Pennsylvania\\
\href{mailto:davidbow@sas.upenn.edu}{\nolinkurl{davidbow@sas.upenn.edu}}}
    \preauthor{\centering\large\emph}
  \postauthor{\par}
      \predate{\centering\large\emph}
  \postdate{\par}
    \date{August 27, 2018}

\usepackage{setspace}
\usepackage{longtable}

\usepackage{float}
\let\origtable\table
\let\endorigtable\endtable
\renewenvironment{table}[1][2] {
    \singlespacing
    \expandafter\origtable\expandafter[H]
} {
    \endorigtable
}

\raggedbottom

\usepackage{amsthm}
\newtheorem{theorem}{Theorem}[section]
\newtheorem{lemma}{Lemma}[section]
\theoremstyle{definition}
\newtheorem{definition}{Definition}[section]
\newtheorem{corollary}{Corollary}[section]
\newtheorem{proposition}{Proposition}[section]
\theoremstyle{definition}
\newtheorem{example}{Example}[section]
\theoremstyle{definition}
\newtheorem{exercise}{Exercise}[section]
\theoremstyle{remark}
\newtheorem*{remark}{Remark}
\newtheorem*{solution}{Solution}
\begin{document}
\maketitle
\begin{abstract}
Recent high-profile examples such as the Free Syrian Army and M23
suggest that when governments violate human rights, they risk spurring
resistance within their own security forces. Does repression generally
lead to regime coups and rebellions originating from the regime? Under
what conditions are we most likely to observe this process? I argue that
when governments engage in repression, they tend to lose legitimacy at
both the domestic and international levels, increasing the risk of
defections from the regime. This risk should be even greater when
significant numbers of soldiers share ethnic ties with the individuals
being repressed, and when the military has limited centralized control
over its members. Using a global sample spanning the years 1946--2013, I
find robust evidence that repression is associated with an increased
probability of coup attempts, and limited evidence for a link to rebel
groups originating from the regime. The ethnic ties hypothesis finds
more support than the military centralization prediction. The results
add support to previous arguments that internal backlash provides a
disincentive for governments to repress.
\end{abstract}

\doublespacing

\setlength{\parindent}{1cm}

\hypertarget{introduction}{%
\section{Introduction}\label{introduction}}

When the Arab Spring protests spread to Syria in March of 2011, the
regime of President Bashar al-Assad quickly responded with forceful
repression, including the torture and killing of a 13-year-old boy
(Macleod and Flamand \protect\hyperlink{ref-Macleod2011}{2011}). While
this brutality was presumably intended to deter threats to the regime,
it arguably backfired by provoking the defection of a substantial
portion of the regime military including Colonel Hussein Harmoush, who
expressed a feeling of complicity in the government actions, saying that
``I defected from the Syrian Arab army and took responsibility for
protecting civilians\ldots{} I feel like I am responsible for the deaths
of every single martyr in Syria,'' (Abouzeid
\protect\hyperlink{ref-Abouzeid2011}{2011}). Riyad al-Assad, who
defected from the Syrian Air to form the Free Syrian Army rebel group,
similarly declared his intent to protect protestors and resist the
regime military (Lister \protect\hyperlink{ref-Lister2016}{2016}).
Several other contemporary rebel groups have similar origins, including
M23 in the Democratic Republic of the Congo, and the National Liberation
Army in Libya, suggesting that the phenomenon could be widespread. This
paper thus seeks to answer two question. First, does the use of
repression place regimes at greater risk of desertion and coups? Second,
under what conditions is repression most likely to produce such
outcomes?

The fragmentation of the regime military is a key dynamic in many
conflicts and potentially explains why the Arab Spring led to civil war
in Syria and Libya, but not in other countries. Similar processes have
occurred in numerous other cases, as original data presented herein
shows that more than 15\% of rebel groups since World War II have traced
their origins to the regime military, and an additional 9\% were founded
by civilian regime officials. These rebellions have received little
attention from scholars, and while some existing theories of civil war
onset such as greed theory (Collier and Hoeffler
\protect\hyperlink{ref-Collier2004}{2004}) could potentially account for
them, many, including theories focusing on ethnic discrimination (e.g.
Cederman, Wimmer, and Min \protect\hyperlink{ref-Cederman2010}{2010})
and protest escalation (e.g. Pierskalla
\protect\hyperlink{ref-Pierskalla2010}{2010}), assume that rebellions
originate outside the government. Studying these cases thus offers the
possibility of enhancing our understanding of civil war onset.

Another prominent form of regime fragmentation --- coups d'etat --- has
been the subject of much scholarship. While most of the existing
literature focuses on broader structural conditions affecting coup risk,
some identify a connection between protests and coup occurrence (Casper
and Tyson \protect\hyperlink{ref-Casper2014}{2014}; Johnson and Thyne
\protect\hyperlink{ref-Johnson2018}{2018}). Hendrix and Salehyan
(\protect\hyperlink{ref-Hendrix2017}{2017}) consider the government's
response to protests, finding that the possibility of regime
fragmentation often deters the use of repression. Yet, fragmentation
does occur, suggesting the need for further research analyzing
fragmentation as a dependent variable. Additionally, deterrence effects
create the possibility of endogeneity, but existing studies have not
fully corrected for this concern. This study advances the literature on
coups and repression by making coups a dependent variable, by accounting
for the possibility of endogeneity through the use of an instrumental
variable, and by examining a wider set of cases than previous studies.

This research also contributes to the literature on human rights.
Understandably, most work in this area has focused on the causes of
human rights violations. Several scholars, however, have turned their
attention to the consequences of human rights violations for outcomes
such as foreign direct investment (Blanton and Blanton
\protect\hyperlink{ref-Blanton2007}{2007}) and foreign aid (Lebovic and
Voeten \protect\hyperlink{ref-Lebovic2009}{2009}), and others have
suggested that repression could provoke infighting amongst regime
factions (Hendrix and Salehyan
\protect\hyperlink{ref-Hendrix2017}{2017}). These consequences of human
rights violations could offer insight to preventing abuses in the
future. For example, if regime fragmentation has the potential to
constrain abusive behavior (Hendrix and Salehyan
\protect\hyperlink{ref-Hendrix2017}{2017}), disrupting the flow of
private benefits to soldiers might undermine solidarity and strengthen
this effect. By comparing specific mechanisms linking repression to
regime fragmentation, this study offers the prospect of such policy
recommendations.

I proceed with a review of the literature on regime fragmentation,
including coups, rebellions, and desertion. Next, I articulate three
theoretical processes that could link repression to regime
fragmentation. I then specify a research design to test these
propositions, and present results from fixed-effects and instrumental
variables regression models. I conclude by situating the results in the
broader literature, and by offering suggestions for future research.

\hypertarget{prior-work-on-regime-fragmentation}{%
\section{Prior Work on Regime
Fragmentation}\label{prior-work-on-regime-fragmentation}}

While some forms of political violence are often considered jointly,
such as protests and civil war or civil and international war, coups
d'etat are typically treated as distinct phenomena and are not
considered jointly with other forms of violence.\footnote{One notable
  exception is Roessler (\protect\hyperlink{ref-Roessler2011}{2011}).}
Given examples such as the Free Syrian Army, however, I argue that there
is considerable overlap between coups and civil wars. Thus, I analyze
coups and rebellions that originate from the regime jointly under the
umbrella term ``regime fragmentation.''

\hypertarget{dissent-repression-and-coups}{%
\subsection{Dissent, Repression, and
Coups}\label{dissent-repression-and-coups}}

The literature often conceptualizes coup attempts as coordination
problems among elites (Weeks \protect\hyperlink{ref-Weeks2008}{2008};
Powell \protect\hyperlink{ref-Powell2012a}{2012}). Protests can
potentially solve this coordination problem by revealing the regime's
ability to deter challenges (Casper and Tyson
\protect\hyperlink{ref-Casper2014}{2014}; Johnson and Thyne
\protect\hyperlink{ref-Johnson2018}{2018}). These signals are likely to
be especially influential when protests are non-violent (Johnson and
Thyne \protect\hyperlink{ref-Johnson2018}{2018}), occur in or near the
national capital (Johnson and Thyne
\protect\hyperlink{ref-Johnson2018}{2018}), and are amplified by a free
press (Casper and Tyson \protect\hyperlink{ref-Casper2014}{2014}).
Protests can also serve as a motive for coup attempts by signaling the
illegitimacy of the regime (Johnson and Thyne
\protect\hyperlink{ref-Johnson2018}{2018}) and enhance opportunities for
successful coups by empowering the military (Svolik
\protect\hyperlink{ref-Svolik2012e}{2012}).

While protests are associated with a statistically-significant increase
in the probability of coups (Casper and Tyson
\protect\hyperlink{ref-Casper2014}{2014}; Johnson and Thyne
\protect\hyperlink{ref-Johnson2018}{2018}), responding to protests with
force is not necessarily a wise choice for regimes. Hendrix and Salehyan
(\protect\hyperlink{ref-Hendrix2017}{2017}) argue that the use of
repression can cause backlash within the military, and show that this
possibility deters repressive tactics, particularly when coup-risk is
especially, as is the case for militaries that have previously
experienced infighting, and for protests which emphasize ethnic or
religious identities. While this deterrent effect is substantial, a
great deal of repression occurs nonetheless. Neither Hendrix and
Salehyan (\protect\hyperlink{ref-Hendrix2017}{2017}) nor any other study
to my knowledge examines the reverse relationship assessing, in effect,
whether the assumption that repression leads to regime fragmentation is
correct. Furthermore, the use of repression is likely endogenous to
potential reactions to its use (Ritter and Conrad
\protect\hyperlink{ref-Ritter2016}{2016}). The body of evidence on the
relationship between repression and regime fragmentation would therefore
be strengthened by analyses using causal identification techniques such
as the instrumental variable analysis presented here.

\hypertarget{repression-and-civil-war}{%
\subsection{Repression and Civil War}\label{repression-and-civil-war}}

Many scholars of political violence and human rights have examined the
``repression-dissent nexus,'' often focusing on the role of government
repression in escalating unrest. There is widespread agreement that
repression can backfire and escalate dissident activities, though the
conditions under which this occurs are contested. Early work in this
area disaggregated dissident activity, showing that repression reduces
non-violent dissent while increasing violent opposition (Lichbach
\protect\hyperlink{ref-Lichbach1987}{1987}; Moore
\protect\hyperlink{ref-Moore1998}{1998}). Rasler
(\protect\hyperlink{ref-Rasler1996}{1996}) emphasizes temporal dynamics,
providing evidence that repression reduces dissident activity in the
short-run while increasing it in the long-run. Pierskalla
(\protect\hyperlink{ref-Pierskalla2010}{2010}) criticizes prior studies
for their lack of attention to strategic interplay and finds that
escalation should only occur in the presence of a third-party threat.
The preceding findings concern primarily the qualitative character of
dissent, rather than the quantitative volume. The consistency of
repressive or accommodative policies (Lichbach
\protect\hyperlink{ref-Lichbach1987}{1987}), the forcefulness of
repressive tactics (Hegre et al.
\protect\hyperlink{ref-Hegre2001}{2001}; Pierskalla
\protect\hyperlink{ref-Pierskalla2010}{2010}), and prior history of
civil conflict (Bell and Murdie \protect\hyperlink{ref-Bell2018}{2018})
have been posited as explanations for the aggregate level of dissent.

The preceding studies generally define escalation in relative terms,
leaving unclear the frequency with which the violence they observe
aligns with scholarly definitions of civil war.\footnote{Such
  definitions typically entail a minimum threshold of severity such as
  25 fatalities per calendar year, a substantial degree of organization
  on both sides, and some amount of competitiveness between the two
  sides. For a representative example, see Pettersson and Eck
  (\protect\hyperlink{ref-Pettersson2018}{2018}).} However, several
works do focus specifically on civil war as a form of escalation. Some
scholars conceptualize repression and civil war as distinct equilibria
(Besley and Persson \protect\hyperlink{ref-Besley2009}{2009}; Choi and
Kim \protect\hyperlink{ref-Choi2018}{2018}). The probability that
repression is met by effective resistance (the civil war outcome) rather
than remaining one-sided is shown to increase with the material value of
winning control of the government (Besley and Persson
\protect\hyperlink{ref-Besley2009}{2009}), decrease with the
inclusiveness of political institutions (Besley and Persson
\protect\hyperlink{ref-Besley2009}{2009}; Choi and Kim
\protect\hyperlink{ref-Choi2018}{2018}), and increase with the size of
the dominant ethnic group relative to the size of the minimum winning
coalition (Choi and Kim \protect\hyperlink{ref-Choi2018}{2018}). Others
treat repression and civil war as sequential steps in a process of
escalation. In this view, repression often proves counterproductive as
it decreases support for the regime (Young
\protect\hyperlink{ref-Young2013}{2013}) and increases the probability
of civil war. Interestingly, this relationship does not appear to be
conditional on the efficacy of repression, as effective repression
increases the risk of civil war by further inflaming tensions, while
ineffective repression does so by emboldening dissidents (Davenport,
Armstrong II, and Lichbach, \protect\hyperlink{ref-Davenportnd}{n.d.}).
In addition to being associated with the onset of civil conflict,
repression predicts increased violence by existing rebel groups
(Shellman, Levey, and Young \protect\hyperlink{ref-Shellman2013}{2013}).

The studies reviewed here have consistently found a link between
repression and the onset of civil conflict. However, tests of specific
theoretical mechanisms have been less conclusive. A potential reason for
this is that while all of the theories above assume that rebel groups
emerge from non-state dissident movements, original data presented here
reveals that a large number of rebellions included in commonly-used
datasets are launched by members of the regime. Differences in the
conditions under which repression produces these two types of rebellions
could account for the inconclusive results on mechanisms observed thus
far. This analysis could shed light on such a possibility.

\hypertarget{theorizing-regime-fragmentation}{%
\section{Theorizing Regime
Fragmentation}\label{theorizing-regime-fragmentation}}

Prior work has demonstrated an empirical link between protests and coup
onset. Repression plays a crucial, but often untested role in the
accompanying theoretical models. The government's response to unrest is
thought to reveal information about its strength, increasing the
probability that potential coup plotters will be able to coordinate
their actions (Casper and Tyson
\protect\hyperlink{ref-Casper2014}{2014}; Johnson and Thyne
\protect\hyperlink{ref-Johnson2018}{2018}). In this view, all else
equal, high levels of repression should reduce the probability of coups
and rebellions by signaling that the government is strong. I argue that
this approach is overly focused on the informational role of protests
and coups, and overlooks the possibility that repression could alter
preferences over challenging the government.

While the use of repression demonstrates a certain degree of military
strength, backlash to its use can weaken the regime in a variety of
ways. Belkin and Schofer (\protect\hyperlink{ref-Belkin2003}{2003})
argue that losing legitimacy in the eyes of the populace is a
prerequisite for coups. As a government becomes more repressive, it
should tend to lose legitimacy and support among citizens. It does not
necessarily follow that regime opponents will experience a concomitant
gain in legitimacy and popularity, particularly if they played a role in
applying the repression. Nevertheless, opponents should gain strength
relative to the regime as pro-government mobilization becomes less
likely. The regime may also lose international support as it engages in
repression. Human rights violations are often met with economic
sanctions, and there is evidence that human rights practices inform
consumer choices (Cao, Greenhill, and Prakash
\protect\hyperlink{ref-Cao2013}{2013}). Although sanctions do not seem
to be effective at deterring human rights abuses (Wood
\protect\hyperlink{ref-Wood2008a}{2008}), they may have dramatic
consequences for internal regime politics. Many regimes maintain the
loyalty of their security force by providing material benefits (Belkin
and Schofer \protect\hyperlink{ref-Belkin2003}{2003}; Powell
\protect\hyperlink{ref-Powell2012a}{2012}), often financed through
non-tax revenue sources such as foreign aid and natural resource rents
(Morrison \protect\hyperlink{ref-Morrison2009}{2009}). If these revenue
streams are disrupted in response to repression, the incentives for
soldiers to continue supporting the regime are reduced.

In short, I expect that repression and the regime's power share a
paradoxical relationship. While the successful use of repression is
indicative of a strong regime, it also tends to set regimes on a path
toward a relative decline in power. Repression should therefore be
positively related to the incidence of coups and regime-based
rebellions.

\noindent \textit{H1a: Repression increases the probability of coups}
\noindent \textit{H1b: Repression increases the probability of regime rebellion}

\hypertarget{ethnic-ties}{%
\subsection{Ethnic Ties}\label{ethnic-ties}}

Ethnicity and religion are the most influential ordering principles in
many societies worldwide (Robinson
\protect\hyperlink{ref-Robinson2014}{2014}), shaping aspects of life
ranging from fundamental tasks such as local-level collective action
(Habyarimana et al. \protect\hyperlink{ref-Habyarimana2007}{2007}) to
broad political patterns such as party preference (Wantchekon
\protect\hyperlink{ref-Wantchekon2003}{2003}). Identity has often been
the basis for protests (Jazayeri
\protect\hyperlink{ref-Jazayeri2016}{2016}; Salehyan and Stewart
\protect\hyperlink{ref-Salehyan2017}{2017}) and violent conflict between
societal groups and the state (Cederman, Gleditsch, and Buhaug
\protect\hyperlink{ref-Cederman2013a}{2013}). This creates many
opportunities for repression, which might be an especially likely
response when a state faces multiple potential identity-based challenges
(Walter \protect\hyperlink{ref-Walter2006a}{2006}).

Soldiers should be especially likely to rebel against the regime when
asked to repress members of their own ethnic group. Evidence suggests
that unusually dense network ties between members are a key reason for
the unique role that ethnic groups often play in politics (Habyarimana
et al. \protect\hyperlink{ref-Habyarimana2007}{2007}). Thus, soldiers
may have some degree of social connection to the co-ethnic targets of
repression, which might reduce their inclination to carry out orders,
and may lead them to leave their posts to defend their families or
communities. Even in the absence of direct social ties, orders to target
co-ethnics may give soldiers pause because ethnic groups are often
defined in part by a sense of ``linked fate'' (Kuran
\protect\hyperlink{ref-Kuran1998}{1998}). Soldiers may fear that a
repressive campaign against dissident members of their ethnic group
could eventually turn against them. Ethnic identity has been shown to be
malleable in response to the political climate (Eifert, Miguel, and
Posner \protect\hyperlink{ref-Eifert2010}{2010}), meaning that even if
soldiers did not identify strongly with their ethnic group previously,
they may begin to do so in response to the repression of co-ethnics.

The preceding logic suggests that the risk of coups and rebellions from
the regime should increase with the proportion of the military that
shares a common identity with the individuals being targeted with
repression. I expect this effect to increase with the severity of
repression. Thus, I hypothesize an interaction effect between the level
of repression and the extent of ethnic ties between the military and the
repressed.

\noindent \textit{H2: There is a positive interaction between the level of repression and the share of the military with ethnic ties to the repressed}

\hypertarget{organizational-control}{%
\subsection{Organizational Control}\label{organizational-control}}

One of the most commonly-cited categories of explanation for coups is
the degree of organizational control within the military. Unsuccessful
coup plotters tend to be punished harshly. The probability of coup
attempts should therefore decrease as the regime's ability to withstand
challenges and apprehend participants. Militaries with high levels of
command and control may even be able to prevent coup attempts altogether
by disrupting the ability of dissatisfied members to coordinate. There
is some disagreement about the conditions under which militaries achieve
such control. Belkin and Schofer
(\protect\hyperlink{ref-Belkin2003}{2003}) view the fragmentation of the
military into multiple branches or factions as an effective form of
coup-proofing, as it reduces the likelihood of successful coordination
amongst a wide swath of the military, while Powell
(\protect\hyperlink{ref-Powell2012a}{2012}) finds that most measures of
factionalism are not significant predictors of coups. One reason for
this discrepancy appears to be that not all intra-military divisions are
created equal, as Hendrix and Salehyan
(\protect\hyperlink{ref-Hendrix2017}{2017}) find indirect evidence that
infighting between military factions increases the risk of coups.
Building on their work, I expect that manifest disunity within the
military in the form of infighting should amplify the effect of
repression. This should be observable as a statistically significant
interaction effect between repression and the occurrence of conflict
within the military.

\noindent \textit{H3: There is a positive interaction between the level of repression and the level of infighting amongst military factions}

\hypertarget{research-design}{%
\section{Research Design}\label{research-design}}

An underlying question in this analysis is whether the regime
fragmentation observed during the Arab Spring was unique to that
historical process, or something that has occurred frequently throughout
time and space. A cross-national quantitative analysis that tests for
such outcomes in a large sample is therefore a logical choice. While
highly disaggregated analyses of political violence are increasingly
common and often facilitate nuanced tests of theoretical mechanisms
(e.g. Hendrix and Salehyan \protect\hyperlink{ref-Hendrix2017}{2017}),
the datasets conducive to this sort of research tend to have limited
spatial and temporal coverage. I thus elect to conduct my analysis at
the country-year level, with a sample covering the entire world over the
period 1946-2017. This provides an initial set of 10,713 cases, though
due to missing data the number used in most analyses is slightly lower.

\hypertarget{dependent-variables}{%
\subsection{Dependent Variables}\label{dependent-variables}}

\emph{Coup Attempt} One form of regime fragmentation is coups d'etat. As
I am primarily interested in the conditions under which members of the
military possess a desire to effect change, I analyze coup attempts
rather than coup success, which depends on a set of somewhat unrelated
factors (Powell \protect\hyperlink{ref-Powell2012a}{2012}). The dataset
comes Powell and Thyne (\protect\hyperlink{ref-Powell2011}{2011}), who
define a coup as an ``illegal and overt attempts by the military or
other elites within the state apparatus to unseat the sitting
executive,'' (Powell and Thyne \protect\hyperlink{ref-Powell2011}{2011},
252). As coup attempts often cluster in space and time, I convert the
coup list to a binary indicator that is equal to 1 for country-years in
which a coup was attempted, and 0 otherwise. This yields 407
country-years with coup attempts.

\emph{Regime-Based Rebellion} Some instances of regime fragmentation are
not included in most coup datasets because they do not work through the
state apparatus. For example, members of the Syrian Air Force turned
against the Assad regime in 2011, but opted to form a rebel group (the
Free Syrian Army) rather than operate through their roles within the
state. I call this phenomenon ``regime-based rebellion,'' and collect
original data to measure it. I begin with a list of the rebel groups
which appear in the Uppsala Conflict Data Program's Dyadic Dataset
(version 18.1) (Pettersson and Eck
\protect\hyperlink{ref-Pettersson2018}{2018}). Inclusion in the dataset
requires that a rebel group was involved in fighting that produced at
least 25 fatalities in at least one calendar year between 1946 and 2017.
I then code the societal origin of the rebel group using a variety of
primary and secondary sources. In cases of limited information, coding
decisions are based on the previous roles of the rebel group's
leader(s). Many rebel groups draw membership from multiple societal
groups; in the present analysis I use only the origin that I deem to be
the largest contributor of personnel. I code a rebel group as
originating from the regime military if its membership is drawn
primarily from individuals who were most recently employed by any branch
of the regime security forces. Rebels who previously held civilian roles
within the government are not included in the category.\footnote{For
  complete coding rules, see Bowden
  (\protect\hyperlink{ref-Bowden2017}{2017}).} From these codings I
construct a binary indicator for country-years in which a regime-based
rebel group was formed. This is the case in 71 country-years. As many
opposition movements employ a diversity of tactics, these overlap
substantially with the coup variable - 46 of the country-years with a
regime-based rebellion also have a coup.

\hypertarget{independent-variables}{%
\subsection{Independent Variables}\label{independent-variables}}

\emph{Latent Protection Score} I use two alternative measures to capture
the concept of repression. The first is Latent Human Protection scores,
version 2 (Fariss \protect\hyperlink{ref-Fariss2014}{2014}; Schnakenberg
and Fariss \protect\hyperlink{ref-Schnakenberg2014}{2014}). Most human
rights datasets are derived from news coverage, creating the possibility
for bias as the depth of coverage and standards against which human
rights practices are evaluated might vary across space and time. To
solve this, the authors combine thirteen of the most prominent scholarly
human rights datasets in a Bayesian measurement model. This produces an
estimate for each country-year based on a mix of the data for that
particular year and the average score for that country and year. The
absolute values of the measure are not inherently meaningful, but range
from roughly -3.1 (most repressive) to 4.7 (most respectful of human
rights). Examples of cases towards the more repressive end of the
spectrum include Saddam Hussein's Iraq, which had a score averaging
around -2.5, and Sudan, which had scores around -3.0 during the genocide
in Darfur.

\emph{NAVCO Repression} The Latent Protection Scores are a useful
measure of the general state of human rights in a given country-year,
but by design smooth over temporary fluctuations in repression levels.
To compensate for this blind spot, I include a direct measure of the
repression of specific protests campaigns from the Nonviolent and
Violent Campaigns and Outcomes (NAVCO) dataset (version 2.0) (Chenoweth
and Lewis
\protect\hyperlink{ref-Chenoweth2013b}{2013}\protect\hyperlink{ref-Chenoweth2013b}{a},
\protect\hyperlink{ref-Chenoweth2013a}{2013}\protect\hyperlink{ref-Chenoweth2013a}{b}).
The dataset includes all known instances of sustained, purposive
campaigns against a government or occupying force between 1946 and 2013.
I utilize the ``repression'' variable, which codes the government
response to a campaign on a four-point scale ranging from 0 (not
repressive) to 3 (extreme repression). When multiple campaigns occur
within a single country-year, I use the most repressive action observed.

\emph{Percentage of Military from Discriminated Ethnic Groups} To test
\emph{Hypothesis 2} I construct a measure from the Security Forces
Ethnicity (SFE) Dataset (version 1.0) (Johnson and Thurber
\protect\hyperlink{ref-Johnson2017}{2017}). The SFE data measures the
ethnic composition of security forces in the Middle East, 1946--2013,
providing approximate ethnic breakdowns of both the officer corps and
rank-and-file. I combine this data with the Ethnic Power Relations (EPR)
Core Dataset 2018 (Vogt et al. \protect\hyperlink{ref-Vogt2015}{2015}),
which codes the political status of each ethnic group worldwide,
1946-2017. I then construct a measure of the percentage of the
rank-and-file membership that belong to an ethnic group facing
discrimination (the ``Discriminated'' or ``Powerless'') codings on the
``status'' variable.

\emph{Military Infighting} To test \emph{Hypothesis 3} I include a
binary indicator of whether conflict between military factions has
occurred. The measure is constructed from the Historical Phoenix Event
Data (version 1.0.0) (Althaus et al.
\protect\hyperlink{ref-Althaus2017}{2017}). Phoenix compiles political
events through automated coding of news articles, recording the actors
involved and classifying the event on the CAMEO events data coding
scheme. I code the indicator as 1 for country-years in which there was
at least one violent conflict between two actors associated with the
government military.

\hypertarget{control-variables}{%
\subsection{Control Variables}\label{control-variables}}

As coups have been shown to be most prevalent in relatively poor states
(Londregan and Poole \protect\hyperlink{ref-Londregan1990}{1990}), I
include the per capita gross domestic product and population using data
from Gleditsch (\protect\hyperlink{ref-Gleditsch2002a}{2002}) (version
6.0 beta) for the period 1950-2011. Both measures are logged. Following
Johnson and Thyne (\protect\hyperlink{ref-Johnson2018}{2018}) I include
binary indicators for autocratic and democratic regimes (leaving
anocracy as the residual category) using data from the Polity IV project
(Marshall, Gurr, and Jaggers
\protect\hyperlink{ref-Marshall2016}{2016}), and a binary indicator for
military regimes using from the Autocratic Regime Data (Geddes, Wright,
and Frantz \protect\hyperlink{ref-Geddes2014a}{2014}). Lastly, I control
for ongoing civil conflicts, including all conflicts producing at least
25 fatalities in a calendar year as reported by the Uppsala Conflict
Data Program Armed Conflict Data (Pettersson and Eck
\protect\hyperlink{ref-Pettersson2018}{2018}).

\hypertarget{model}{%
\subsection{Model}\label{model}}

As both dependent variables are binary, I estimate panel logistic
regression models. Because I am primarily interested in the effect of
changes within countries, I use a fixed-effects estimator. To help
ensure that the explanatory variables cause regime fragmentation rather
than the reverse, all independent variables and controls are lagged by
one year. For robustness I estimate instrumental variables probit
models, discussed in detail below.

\hypertarget{results}{%
\section{Results}\label{results}}

Results for \emph{Hypotheses 1a} and \emph{1b} are presented in Table 1.
Consistent with my predictions, repression generally predicts the onset
of coups and regime-based rebellions. Model 1 uses the \emph{Latent
Human Protection Score} to measure repression, with coup attempt as the
dependent variable. Higher values of the protection score signify
\emph{greater} levels of respect for human rights; I thus expect a
negative relationship between this variable and coups. Indeed, the
measure has a negative relationship with coup probability that is
statistically significant at the 99.9\% level. The effect is
substantively large, as a one-unit decrease in the \emph{Latent
Protection Score} (for reference, this was roughly the difference
between Israel and the Democratic Republic of the Congo in 2013)
increases the probability of a coup by 64.8\%. Model 2 shows the
relationship between \emph{Latent Protection Score} and the onset of
regime-based rebellion. Once again consistent with my prediction, the
relationship is negative and statistically significant. The substantive
effect is even larger, with a one-unit decrease in the protection score
increasing the probability of rebellion by 156\%. The number of
observations decreases by more than half for the regime rebellion
dependent variable, as many countries never experience the phenomenon
and fixed-effects estimation requires variation on the dependent
variable.

The effects of the \emph{NAVCO Repression} variable are reported in
Models 3 and 4. In this cases higher values are associated with greater
levels of repression, meaning that I would expect as positive
coefficient. The result for coup attempts is consistent with my
prediction, as \emph{NAVCO Repression} has a positive coefficient that
is statistically significant at the 95\% level. The effect size is
modest in comparison to that of \emph{Latent Protection Score}, as a
one-unit increase in repression (for example, movement from ``moderate''
to ``extreme'' repression) increases the probability of a coup by
17.4\%. \emph{NAVCO Repression} is not significantly related to
regime-based rebellion. On the whole, these results suggest strong
support for Hypotheses 1a and 1b --- repression is a strong predictor of
regime fragmentation, manifest in both coup attempts and rebellions
originating from the regime.

\singlespacing

\begin{table}[htbp]\centering
\def\sym#1{\ifmmode^{#1}\else\(^{#1}\)\fi}
\caption{\label{tab1} Fixed-Effects Logit Models of Coups}
\begin{tabular}{l*{4}{c}}
\hline\hline
                    &\multicolumn{1}{c}{(1)}         &\multicolumn{1}{c}{(2)}         &\multicolumn{1}{c}{(3)}         &\multicolumn{1}{c}{(4)}         \\
                    &        Coup         &Regime Rebellion         &        Coup         &Regime Rebellion         \\
\hline
Latent Protection Score&       -0.50\sym{***}&       -0.94\sym{***}&                     &                     \\
                    &      (0.11)         &      (0.23)         &                     &                     \\
NAVCO Repression    &                     &                     &        0.16\sym{*}  &        0.25         \\
                    &                     &                     &      (0.07)         &      (0.14)         \\
Autocracy           &       -0.65\sym{***}&       -0.85\sym{**} &       -0.52\sym{***}&       -0.58         \\
                    &      (0.15)         &      (0.33)         &      (0.15)         &      (0.32)         \\
Democracy           &       -0.24         &       -0.19         &       -0.31         &       -0.37         \\
                    &      (0.20)         &      (0.48)         &      (0.20)         &      (0.48)         \\
Military Regime     &        0.33         &        0.17         &        0.37\sym{*}  &        0.34         \\
                    &      (0.18)         &      (0.46)         &      (0.18)         &      (0.45)         \\
log GDPpc           &       -0.24         &        0.08         &       -0.30         &       -0.05         \\
                    &      (0.17)         &      (0.40)         &      (0.17)         &      (0.38)         \\
log Population      &       -1.54\sym{***}&       -1.01\sym{*}  &       -1.35\sym{***}&       -0.58         \\
                    &      (0.18)         &      (0.40)         &      (0.17)         &      (0.36)         \\
Civil Conflict      &        0.22         &       -0.98\sym{*}  &        0.32         &       -0.67         \\
                    &      (0.19)         &      (0.42)         &      (0.20)         &      (0.45)         \\
\hline
N                   &        4340         &        2210         &        4340         &        2210         \\
\hline\hline
\multicolumn{5}{l}{\footnotesize \sym{*} \(p<0.05\), \sym{**} \(p<0.01\), \sym{***} \(p<0.001\)}\\
\end{tabular}
\end{table}


\doublespacing

Results for \emph{H2} are presented in Table 2. The first column reveals
an statistically significant interaction between \emph{Latent Protection
Score} and \emph{\% Military Discriminated}. The marginal effects
plotted in Figure 1 reveal how the variables affect each other.
\emph{Latent Protection Score} is virtually unaffected by the level of
\emph{\% Military Discriminated}, having a consistent, slightly positive
effect on coup probability. This pattern contradicts my expectations
(recall that higher levels of the measure equate to \emph{less}
repression). The effect of \emph{\% Military Discriminated}, however, is
highly conditional on the level of \emph{Latent Protection Score}. At
the most repressive end of the scale, the share of the military that
belongs to a marginalized ethnic group is positively related to the
probability of a coup. At the least repressive end of the spectrum,
however, discrimination of ethnic groups well-represented in the
military is \emph{negatively} related to the probability of coups. One
interpretation of these result is that the violent repression of
co-ethnics induces soldiers to turn against the regime, while
non-violent forms of discrimination do not. It may also be the case that
the different repressive strategies (violent vs.~non-violent) correlate
with a confounding factor such as coup-proofing efforts. The former
explanation would be consistent with my theoretical framework, while the
latter likely would not be. The \emph{\% Military Discriminated} measure
is not statistically significant in a simple additive model, and neither
the interactive nor the additive effect is significantly related to
regime-based rebellion.

\singlespacing

\begin{table}[htbp]\centering
\def\sym#1{\ifmmode^{#1}\else\(^{#1}\)\fi}
\caption{\label{tab2} The Effect of Military Ethnic Ties and Repression on Regime Fragmentation}
\begin{tabular}{l*{4}{c}}
\hline\hline
                    &\multicolumn{1}{c}{(1)}         &\multicolumn{1}{c}{(2)}         &\multicolumn{1}{c}{(3)}         &\multicolumn{1}{c}{(4)}         \\
                    &        Coup         &        Coup         &Regime Rebellion         &Regime Rebellion         \\
\hline
Latent Protection   &        0.06         &       -0.58\sym{*}  &       -1.20         &       -0.88         \\
Score               &      (0.35)         &      (0.29)         &      (0.68)         &      (0.61)         \\
\% Military         &       -0.04\sym{**} &       -0.00         &        0.00         &       -0.03         \\
Discriminated       &      (0.01)         &      (0.01)         &      (0.03)         &      (0.02)         \\
Autocracy           &       -0.78         &       -0.53         &        1.22         &        0.69         \\
                    &      (0.41)         &      (0.40)         &      (1.33)         &      (1.10)         \\
Democracy           &       -0.35         &       -0.21         &                     &                     \\
                    &      (0.60)         &      (0.62)         &                     &                     \\
Military Regime     &       -0.79         &       -0.66         &       -0.59         &       -0.42         \\
                    &      (0.57)         &      (0.57)         &      (1.47)         &      (1.36)         \\
log GDPpc           &       -0.07         &       -0.06         &        0.67         &        0.40         \\
                    &      (0.41)         &      (0.39)         &      (1.11)         &      (1.08)         \\
log Population      &       -1.74\sym{***}&       -1.74\sym{***}&       -0.89         &       -0.53         \\
                    &      (0.47)         &      (0.44)         &      (1.31)         &      (1.18)         \\
Civil Conflict      &       -0.34         &       -0.04         &       -1.05         &       -1.20         \\
                    &      (0.45)         &      (0.43)         &      (0.95)         &      (0.90)         \\
Latent Protection   &       -0.03\sym{**} &                     &        0.02         &                     \\
Score $\times$ \% Military Discriminated&      (0.01)         &                     &      (0.02)         &                     \\
\hline
N                   &         778         &         778         &         269         &         269         \\
\hline\hline
\multicolumn{5}{l}{\footnotesize \sym{*} \(p<0.05\), \sym{**} \(p<0.01\), \sym{***} \(p<0.001\)}\\
\end{tabular}
\end{table}


\doublespacing

\begin{figure}
\centering
\includegraphics{/Users/david/Dropbox/defection_manuscript/data/interaction1_2_combined.pdf}
\caption{Marginal Effects of Repression x Military Discrimination
Interaction}
\end{figure}

The results for \emph{H3} are presented in Table 3. The interaction of
\emph{Military Infighting} and \emph{Latent Protection Score} is
significant for neither coup attempts nor regime-based rebellions. The
infighting measure is also not significant in the additive models. In
fact, the only noteworthy result in Table 3 is that the effect of
\emph{Latent Protection Score} is robust to the inclusion of the
\emph{Military Infighting} measure.

\singlespacing

\begin{table}[htbp]\centering
\def\sym#1{\ifmmode^{#1}\else\(^{#1}\)\fi}
\caption{\label{tab3} The Effect of Military Infighting and Repression on Regime Fragmentation}
\begin{tabular}{l*{4}{c}}
\hline\hline
                    &\multicolumn{1}{c}{(1)}         &\multicolumn{1}{c}{(2)}         &\multicolumn{1}{c}{(3)}         &\multicolumn{1}{c}{(4)}         \\
                    &        Coup         &        Coup         &Regime Rebellion         &Regime Rebellion         \\
\hline
Latent Protection   &       -0.50\sym{***}&       -0.53\sym{***}&       -0.94\sym{***}&       -0.91\sym{***}\\
Score               &      (0.11)         &      (0.11)         &      (0.23)         &      (0.23)         \\
Military Infighting &       -0.11         &        0.09         &       -0.04         &       -0.43         \\
                    &      (0.19)         &      (0.25)         &      (0.40)         &      (0.66)         \\
Autocracy           &       -0.65\sym{***}&       -0.65\sym{***}&       -0.85\sym{**} &       -0.86\sym{**} \\
                    &      (0.15)         &      (0.15)         &      (0.33)         &      (0.33)         \\
Democracy           &       -0.25         &       -0.25         &       -0.19         &       -0.18         \\
                    &      (0.20)         &      (0.20)         &      (0.48)         &      (0.48)         \\
Military Regime     &        0.33         &        0.33         &        0.17         &        0.17         \\
                    &      (0.18)         &      (0.18)         &      (0.46)         &      (0.46)         \\
log GDPpc           &       -0.24         &       -0.25         &        0.08         &        0.10         \\
                    &      (0.17)         &      (0.17)         &      (0.40)         &      (0.40)         \\
log Population      &       -1.53\sym{***}&       -1.52\sym{***}&       -1.01\sym{*}  &       -1.02\sym{*}  \\
                    &      (0.18)         &      (0.18)         &      (0.41)         &      (0.41)         \\
Civil Conflict      &        0.23         &        0.24         &       -0.97\sym{*}  &       -1.00\sym{*}  \\
                    &      (0.19)         &      (0.19)         &      (0.42)         &      (0.43)         \\
Latent Protection   &                     &        0.22         &                     &       -0.34         \\
Score $\times$ Military Infighting&                     &      (0.20)         &                     &      (0.44)         \\
\hline
N                   &        4340         &        4340         &        2210         &        2210         \\
\hline\hline
\multicolumn{5}{l}{\footnotesize \sym{*} \(p<0.05\), \sym{**} \(p<0.01\), \sym{***} \(p<0.001\)}\\
\end{tabular}
\end{table}


\doublespacing

\hypertarget{causal-inference}{%
\section{Causal Inference}\label{causal-inference}}

The results presented thus far are consistent with \emph{H1a} and
\emph{H1b}, but face a notable shortcoming --- a government's decision
to repress or not may be endogenous to the potential consequences of
repression. Indeed, Hendrix and Salehyan
(\protect\hyperlink{ref-Hendrix2017}{2017}) argue that the risk of
internal backlash deters repression. Instrumental variables have been
established as an effective means of correcting for such bias (Ritter
and Conrad \protect\hyperlink{ref-Ritter2016}{2016}). Valid instruments
are often difficult to find, as they must be strong predictors of the
endogenous variable (repression in this case), but cannot be related to
the dependent variable except through the effects of the endogenous
variable. One variable that may satisfy both requirements is the ``youth
bulge'' measure originally proposed by Urdal
(\protect\hyperlink{ref-Urdal2006}{2006}). Youth bulges are a
national-level demographic attribute defined as the ratio of 15-24
year-olds relative to the adult population. In a meta analysis Hill and
Jones (\protect\hyperlink{ref-Hill2014}{2014}) show that the youth bulge
is among the strongest predictors of repression in the literature,
satisfying the first requirement of an instrument. It is less clear
whether youth bulges meet the second require (the exclusion
restriction). On one hand Urdal
(\protect\hyperlink{ref-Urdal2006}{2006}) finds that youth bulges are
strongly associated with political violence. On the other hand, the
theoretical mechanisms he outlines emphasize challenges that originate
outside the state, motivated by resource shortages. We should not
necessarily expect, then, that youth bulges would have a direct effect
on defection from the regime. I thus proceed to utilize the measure as
an instrument in a two-stage probit model.

\singlespacing

\begin{table}[htbp]\centering
\def\sym#1{\ifmmode^{#1}\else\(^{#1}\)\fi}
\caption{\label{tab4} IV Probit Models of the Effect of Repression on Regime Fragmentation}
\begin{tabular}{l*{3}{c}}
\hline\hline
                    &\multicolumn{1}{c}{(1)}         &\multicolumn{1}{c}{(2)}         &\multicolumn{1}{c}{(3)}         \\
                    &        Coup         &        Coup         &Regime Rebellion         \\
\hline
Latent Protection Score&       -0.30         &                     &       -0.38         \\
                    &      (0.23)         &                     &      (0.30)         \\
NAVCO Repression    &                     &        1.21\sym{***}&                     \\
                    &                     &      (0.31)         &                     \\
Autocracy           &       -0.34\sym{***}&       -0.12         &       -0.19         \\
                    &      (0.09)         &      (0.16)         &      (0.13)         \\
Democracy           &       -0.24         &       -0.17         &       -0.14         \\
                    &      (0.19)         &      (0.23)         &      (0.31)         \\
Military Regime     &        0.52\sym{***}&       -0.21         &        0.16         \\
                    &      (0.12)         &      (0.45)         &      (0.17)         \\
log GDPpc           &       -0.16         &       -0.03         &       -0.11         \\
                    &      (0.09)         &      (0.15)         &      (0.09)         \\
log Population      &       -0.15\sym{*}  &       -0.14\sym{***}&       -0.08         \\
                    &      (0.06)         &      (0.04)         &      (0.08)         \\
Civil Conflict      &       -0.11         &       -1.81\sym{**} &       -0.32         \\
                    &      (0.26)         &      (0.61)         &      (0.36)         \\
Constant            &        1.06\sym{*}  &        0.68         &       -0.72         \\
                    &      (0.42)         &      (0.65)         &      (0.43)         \\
\hline
N                   &        7595         &        7595         &        7595         \\
\hline\hline
\multicolumn{4}{l}{\footnotesize \sym{*} \(p<0.05\), \sym{**} \(p<0.01\), \sym{***} \(p<0.001\)}\\
\end{tabular}
\end{table}


\doublespacing

The results of the instrumental variables analysis are presented in
Table 4. In the first and third models, the youth bulge measure is used
to instrument for \emph{Latent Protection Score}. The statistically
significant relationship exhibited by the direct disappears after
instrumenting. When instrumenting for the \emph{NAVCO Repression}
measure, however, the relationship remains statistically significant and
is larger, substantively than in the conventional regression.\footnote{It
  was not possible to obtain estimates for the regime rebellion
  dependent variable.} The instrumental variables analysis thus produces
ambiguous results. One explanation is that by including a wide variety
of data sources, the \emph{Latent Protection Scores} capture information
that is endogenous to the dependent variables studied here. It may even
be the case that coups are factoring into the scores directly, though it
is not clear through which underlying dataset this would occur. The
\emph{NAVCO} measure, by contrast, is more narrowly focused on the
repression of non-state political movements, and thus less beset by the
concerns.

\hypertarget{conclusion}{%
\section{Conclusion}\label{conclusion}}

Anecdotally, repression seems to engender backlash and increase the risk
of regime fragmentation. The evidence presented here suggests that cases
like the Free Syrian Army and M23 are not rare. Repression is robustly
related to coup attempts, and there is some evidence to suggest that it
is an important cause of rebellions originating from the regime. The
search for specific mechanisms found limited success, but it does appear
that the combination of high levels of repression and ethnic ties
between members of the military and the individuals being repressed
increases coup risk.

These results have two important implications for the existing
literature. First, they lend support to the argument of Hendrix and
Salehyan (\protect\hyperlink{ref-Hendrix2017}{2017}) that regimes face
potential internal backlash over the use of repression. However, this
study also shows that repression and subsequent fragmentation frequently
do occur. Thus, the deterrence effect proposed by Hendrix and Salehyan
(\protect\hyperlink{ref-Hendrix2017}{2017}) is not perfectly effective,
leaving a need for further research on the conditions under which
repression does and does not provoke a backlash. These findings also
have implications for the coup literature. Theories of coups often
assume they are opportunistic and motivated by material greed. While
there could be a material link between repression and regime
fragmentation (see the discussion in the theory section regarding
non-tax revenue), the finding that repression strongly predicts coups
could also suggest that coups are guided by more ideational and moral
concerns than previously recognized.

\hypertarget{references}{%
\section*{References}\label{references}}
\addcontentsline{toc}{section}{References}

\markboth{REFERENCES}{}

\indent

\setlength{\parindent}{-0.2in}
\setlength{\leftskip}{0.2in}
\setlength{\parskip}{8pt}

\singlespacing

\hypertarget{refs}{}
\leavevmode\hypertarget{ref-Abouzeid2011}{}%
Abouzeid, Rania. 2011. ``The Soldier Who Gave up on Assad to Protect
Syria's People.'' \emph{Time}.
https://content.time.com/time/world/article/0,8599,2077348,00.html.

\leavevmode\hypertarget{ref-Althaus2017}{}%
Althaus, Scott, Joseph Bajjalieh, John F. Carter, Buddy Peyton, and Dan
A. Shalmon. 2017. ``Cline Center Historical Phoenix Event Data.
V.1.0.0.'' Distributed by Cline Center for Advanced Social Research.

\leavevmode\hypertarget{ref-Belkin2003}{}%
Belkin, Aaron, and Evan Schofer. 2003. ``Toward a Structural
Understanding of Coup Risk.'' \emph{Journal of Conflict Resolution} 47
(5): 594--620.

\leavevmode\hypertarget{ref-Bell2018}{}%
Bell, Sam R, and Amanda Murdie. 2018. ``The Apparatus for Violence:
Repression, Violent Protest, and Civil War in a Cross-National
Framework.'' \emph{Conflict Management and Peace Science} 35 (4):
336--54.

\leavevmode\hypertarget{ref-Besley2009}{}%
Besley, Timothy, and Torsten Persson. 2009. ``Repression or Civil War?''
\emph{American Economic Review} 99 (2): 292--97.

\leavevmode\hypertarget{ref-Blanton2007}{}%
Blanton, Shannon Lindsey, and Robert G. Blanton. 2007. ``What Attracts
Foreign Investors? An Examination of Human Rights and Foreign Direct
Investment.'' \emph{The Journal of Politics} 69 (1): 143--55.

\leavevmode\hypertarget{ref-Bowden2017}{}%
Bowden, David F. 2017. ``Politics Among Rebels: The Causes of Division
Between Dissidents.'' Ph.D. Diss., University of Illinois,
Urbana-Champaign.

\leavevmode\hypertarget{ref-Cao2013}{}%
Cao, Xun, Brian Greenhill, and Aseem Prakash. 2013. ``Where Is the
Tipping Point? Bilateral Trade and the Diffusion of Human Rights.''
\emph{British Journal of Political Science} 43 (1): 133--56.

\leavevmode\hypertarget{ref-Casper2014}{}%
Casper, Brett Allen, and Scott A. Tyson. 2014. ``Popular Protest and
Elite Coordination in a Coup d'état.'' \emph{Journal of Politics} 76
(2): 548--64.

\leavevmode\hypertarget{ref-Cederman2013a}{}%
Cederman, Lars-Erik, Kristian Skrede Gleditsch, and Halvard Buhaug.
2013. \emph{Inequality, Grievances, and Civil War}. New York: Cambridge
University Press.

\leavevmode\hypertarget{ref-Cederman2010}{}%
Cederman, Lars-Erik, Andreas Wimmer, and Brian Min. 2010. ``Why Do
Ethnic Groups Rebel?: New Data and Analysis.'' \emph{World Politics} 62
(1): 87--98.

\leavevmode\hypertarget{ref-Chenoweth2013b}{}%
Chenoweth, Erica, and Orion A. Lewis. 2013a. ``Nonviolent and Violent
Campaigns and Outcomes Dataset, V. 2.0.'' University of Denver.

\leavevmode\hypertarget{ref-Chenoweth2013a}{}%
Chenoweth, Erica, and Orion A Lewis. 2013b. ``Unpacking Nonviolent
Campaigns: Introducing the NAVCO 2.0 Dataset.'' \emph{Journal of Peace
Research} 50 (3): 415--23.

\leavevmode\hypertarget{ref-Choi2018}{}%
Choi, Hyun Jin, and Dongsuk Kim. 2018. ``Coup, Riot, War: How Political
Institutions and Ethnic Politics Shape Alternative Forms of Political
Violence.'' \emph{Terrorism and Political Violence} 30 (4): 718--39.

\leavevmode\hypertarget{ref-Collier2004}{}%
Collier, Paul, and Anke Hoeffler. 2004. ``Greed and Grievance in Civil
War.'' \emph{Oxford Economic Papers} 56 (4): 563--95.

\leavevmode\hypertarget{ref-Davenportnd}{}%
Davenport, Christian, David A. Armstrong II, and Mark I. Lichbach. n.d.
``From Mountains to Movements : Dissent , Repression and Escalation to
Civil War.'' \emph{Unpublished}.

\leavevmode\hypertarget{ref-Eifert2010}{}%
Eifert, Benn, Edward Miguel, and Daniel N. Posner. 2010. ``Political
Competition and Ethnic Identification in Africa.'' \emph{American
Journal of Political Science} 54 (2): 494--510.

\leavevmode\hypertarget{ref-Fariss2014}{}%
Fariss, Christopher J. 2014. ``Respect for Human Rights Has Improved
over Time: Modeling the Changing Standard of Accountability.''
\emph{American Political Science Review} 108 (2): 297--318.

\leavevmode\hypertarget{ref-Geddes2014a}{}%
Geddes, Barbara, Joseph Wright, and Erica Frantz. 2014. ``Autocratic
Breakdown and Regime Transitions: A New Data Set.'' \emph{Perspectives
on Politics} 12 (02): 313--31.

\leavevmode\hypertarget{ref-Gleditsch2002a}{}%
Gleditsch, Kristian Skrede. 2002. ``Expanded Trade and GDP Data.''
\emph{Journal of Conflict Resolution} 46 (5): 712--24.

\leavevmode\hypertarget{ref-Habyarimana2007}{}%
Habyarimana, James, Macartan Humphreys, Daniel N. Posner, and Jeremy M.
Weinstein. 2007. ``Why Does Ethnic Diversity Undermine Public Goods
Provision ?'' \emph{American Political Science Review} 101 (4): 709--25.

\leavevmode\hypertarget{ref-Hegre2001}{}%
Hegre, Havard, Tanja Ellison, Scott Gates, and Nils Petter Gleditsch.
2001. ``Toward a Democratic Civil Peace? Democracy, Political Change,
and Civil War, 1816-1992.'' \emph{American Political Science Review} 95
(1): 33--48.

\leavevmode\hypertarget{ref-Hendrix2017}{}%
Hendrix, Cullen S., and Idean Salehyan. 2017. ``A House Divided: Threat
Perception, Military Factionalism, and Repression in Africa.''
\emph{Journal of Conflict Resolution} 61 (8): 1653--81.

\leavevmode\hypertarget{ref-Hill2014}{}%
Hill, Daniel W, and Zachary M Jones. 2014. ``An Empirical Evaluation of
Explanations for State Repression'' 108 (3): 661--87.

\leavevmode\hypertarget{ref-Jazayeri2016}{}%
Jazayeri, Karen Bodnaruk. 2016. ``Identity-Based Political Inequality
and Protest: The Dynamic Relationship Between Political Power and
Protest in the Middle East and North Africa.'' \emph{Conflict Management
and Peace Science} 33 (4): 400--422.

\leavevmode\hypertarget{ref-Johnson2018}{}%
Johnson, Jaclyn, and Clayton L. Thyne. 2018. ``Squeaky Wheels and Troop
Loyalty: How Domestic Protests Influence Coups d'état, 19512005.''
\emph{Journal of Conflict Resolution} 62 (3): 597--625.

\leavevmode\hypertarget{ref-Johnson2017}{}%
Johnson, Paul Lorenzo, and Ches Thurber. 2017. ``The Security-Force
Ethnicity (SFE) Project: Introducing a New Dataset.'' \emph{Conflict
Management and Peace Science}, 073889421770901.

\leavevmode\hypertarget{ref-Kuran1998}{}%
Kuran, Timur. 1998. ``Ethnic Dissimilation and Its International
Diffusion.'' In \emph{The International Spread of Ethnic Conflict: Fear,
Diffusion, and Escalation}, edited by David A. Lake and Donald
Rothchild, 35--60. Princeton, NJ: Princeton University Press.

\leavevmode\hypertarget{ref-Lebovic2009}{}%
Lebovic, James H., and Erik Voeten. 2009. ``The Cost of Shame:
International Organizations and Foreign Aid in the Punishing of Human
Rights Violators.'' \emph{Journal of Peace Research} 46 (1): 79--97.

\leavevmode\hypertarget{ref-Lichbach1987}{}%
Lichbach, Mark Irving. 1987. ``Deterrence or Escalation? The Puzzle of
Aggregate Studies of Repression and Dissent.'' \emph{Journal of Conflict
Resolution} 31 (2): 266--97.

\leavevmode\hypertarget{ref-Lister2016}{}%
Lister, Charles. 2016. ``The Free Syrian Army: A Decentralized Insurgent
Brand.'' \emph{Brookings Project on U.S. Relations with the Islamic
World}, no. 26.

\leavevmode\hypertarget{ref-Londregan1990}{}%
Londregan, John B., and Keith T. Poole. 1990. ``Poverty, the Coup Trap,
and the Seizure of Executive Power.'' \emph{World Politics} 42 (2):
151--83.

\leavevmode\hypertarget{ref-Macleod2011}{}%
Macleod, Hugh, and Annasofie Flamand. 2011. ``Tortured and Killed: Hamza
Al-Khateeb, Age 13.'' \emph{Al Jazeera}.
https://www.aljazeera.com/indepth/features/2011/05/201153185927813389.html.

\leavevmode\hypertarget{ref-Marshall2016}{}%
Marshall, Monty G., Ted Robert Gurr, and Keith Jaggers. 2016. ``Polity
IV Project Dataset Users' Manual, V.2015.'' \emph{Polity IV Project},
1--86.

\leavevmode\hypertarget{ref-Moore1998}{}%
Moore, Will H. 1998. ``Repression and Dissent: Substitution, Context,
and Timing.'' \emph{American Journal of Political Science} 42 (3):
851--73.

\leavevmode\hypertarget{ref-Morrison2009}{}%
Morrison, Kevin M. 2009. ``Oil, Nontax Revenue, and the Redistributional
Foundations of Regime Stability.'' \emph{International Organization} 63
(1): 107--38.

\leavevmode\hypertarget{ref-Pettersson2018}{}%
Pettersson, Therése, and Kristine Eck. 2018. ``Organized Violence,
1989-2017.'' \emph{Journal of Peace Research} 55 (4): 535--47.

\leavevmode\hypertarget{ref-Pierskalla2010}{}%
Pierskalla, Jan Henryk. 2010. ``Protest, Deterrence, and Escalation: The
Strategic Calculus of Government Repression.'' \emph{Journal of Conflict
Resolution} 54 (1): 117--45.

\leavevmode\hypertarget{ref-Powell2012a}{}%
Powell, Jonathan. 2012. ``Determinants of the Attempting and Outcome of
Coups d'état.'' \emph{Journal of Conflict Resolution} 56 (6): 1017--40.

\leavevmode\hypertarget{ref-Powell2011}{}%
Powell, Jonathan M., and Clayton L. Thyne. 2011. ``Global Instances of
Coups from 1950 to 2010: A New Dataset.'' \emph{Journal of Peace
Research} 48 (2): 249--59.

\leavevmode\hypertarget{ref-Rasler1996}{}%
Rasler, Karen. 1996. ``Concessions, Repression, and Political Protest in
the Iranian Revolution.'' \emph{American Sociological Review} 61 (1):
132.

\leavevmode\hypertarget{ref-Ritter2016}{}%
Ritter, Emily Hencken, and Courtenay R. Conrad. 2016. ``Preventing and
Responding to Dissent: The Observational Challenges of Explaining
Strategic Repression.'' \emph{American Political Science Review} 110
(1): 85--99.

\leavevmode\hypertarget{ref-Robinson2014}{}%
Robinson, Amanda Lea. 2014. ``National Versus Ethnic Identification in
Africa: Modernization, Colonial Legacy, and the Origins of Territorial
Nationalism.'' \emph{World Politics} 66 (4): 709--46.

\leavevmode\hypertarget{ref-Roessler2011}{}%
Roessler, Philip. 2011. ``The Enemy Within: Personal Rule, Coups, and
Civil War in Africa.'' \emph{World Politics} 63 (2): 300--346.

\leavevmode\hypertarget{ref-Salehyan2017}{}%
Salehyan, Idean, and Brandon Stewart. 2017. ``Political Mobilization and
Government Targeting: When Do Dissidents Challenge the State?''
\emph{Comparative Political Studies} 50 (7): 963--91.

\leavevmode\hypertarget{ref-Schnakenberg2014}{}%
Schnakenberg, Keith E., and Christopher J. Fariss. 2014. ``Dynamic
Patterns of Human Rights Practices.'' \emph{Political Science Research
and Methods} 2 (1): 1--31.

\leavevmode\hypertarget{ref-Shellman2013}{}%
Shellman, Stephen M., Brian P. Levey, and Joseph K. Young. 2013.
``Shifting Sands: Explaining and Predicting Phase Shifts by Dissident
Organizations.'' \emph{Journal of Peace Research} 50 (3): 319--36.

\leavevmode\hypertarget{ref-Svolik2012e}{}%
Svolik, Milan W. 2012. ``Contracting on Violence: The Moral Hazard in
Authoritarian Repression and Military Intervention in Politics.''
\emph{Journal of Conflict Resolution} 57 (5): 765--94.

\leavevmode\hypertarget{ref-Urdal2006}{}%
Urdal, Henrik. 2006. ``A Clash of Generations? Youth Bulges and
Political Violence.'' \emph{International Studies Quarterly} 50 (3):
607--29.

\leavevmode\hypertarget{ref-Vogt2015}{}%
Vogt, Manuel, Nils-Christian Bormann, Seraina Ruegger, Lars-Erik
Cederman, Philipp Hunziker, and Luc Girardin. 2015. ``Integrating Data
on Ethnicity, Geography, and Conflict: The Ethnic Power Relations Data
Set Family.'' \emph{Journal of Conflict Resolution} 59 (7): 1327--42.

\leavevmode\hypertarget{ref-Walter2006a}{}%
Walter, Barbara F. 2006. ``Building Reputation: Why Governments Fight
Some Separatists but Not Others.'' \emph{American Journal of Political
Science} 50 (2): 313--30.

\leavevmode\hypertarget{ref-Wantchekon2003}{}%
Wantchekon, Leonard. 2003. ``Clientelism and Voting Behavior: Evidence
from a Field Experiment in Benin.'' \emph{World Politics} 55 (3):
399--422.

\leavevmode\hypertarget{ref-Weeks2008}{}%
Weeks, Jessica L. 2008. ``Autocratic Audience Costs: Regime Type and
Signaling Resolve.'' \emph{International Organization} 62 (1): 35--64.

\leavevmode\hypertarget{ref-Wood2008a}{}%
Wood, Elisabeth Jean. 2008. ``The Social Processes of Civil War: The
Wartime Transformation of Social Networks.'' \emph{Annual Review of
Political Science} 11 (1): 539--61.

\leavevmode\hypertarget{ref-Young2013}{}%
Young, Joseph K. 2013. ``Repression, Dissent, and the Onset of Civil
War.'' \emph{Political Research Quarterly} 66 (3): 516--32.


\end{document}
