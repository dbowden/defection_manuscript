\documentclass[12pt,]{article}
\usepackage[]{tgpagella}
\usepackage{amssymb,amsmath}
\usepackage{ifxetex,ifluatex}
\usepackage{fixltx2e} % provides \textsubscript
\ifnum 0\ifxetex 1\fi\ifluatex 1\fi=0 % if pdftex
  \usepackage[T1]{fontenc}
  \usepackage[utf8]{inputenc}
\else % if luatex or xelatex
  \ifxetex
    \usepackage{mathspec}
  \else
    \usepackage{fontspec}
  \fi
  \defaultfontfeatures{Ligatures=TeX,Scale=MatchLowercase}
\fi
% use upquote if available, for straight quotes in verbatim environments
\IfFileExists{upquote.sty}{\usepackage{upquote}}{}
% use microtype if available
\IfFileExists{microtype.sty}{%
\usepackage{microtype}
\UseMicrotypeSet[protrusion]{basicmath} % disable protrusion for tt fonts
}{}
\usepackage[margin=1in]{geometry}
\usepackage{hyperref}
\hypersetup{unicode=true,
            pdftitle={I Didn't Sign Up for This: Repression and the Fragmentation of Regime Forces},
            pdfauthor={David Bowden},
            pdfborder={0 0 0},
            breaklinks=true}
\urlstyle{same}  % don't use monospace font for urls
\usepackage{longtable,booktabs}
\usepackage{graphicx,grffile}
\makeatletter
\def\maxwidth{\ifdim\Gin@nat@width>\linewidth\linewidth\else\Gin@nat@width\fi}
\def\maxheight{\ifdim\Gin@nat@height>\textheight\textheight\else\Gin@nat@height\fi}
\makeatother
% Scale images if necessary, so that they will not overflow the page
% margins by default, and it is still possible to overwrite the defaults
% using explicit options in \includegraphics[width, height, ...]{}
\setkeys{Gin}{width=\maxwidth,height=\maxheight,keepaspectratio}
\IfFileExists{parskip.sty}{%
\usepackage{parskip}
}{% else
\setlength{\parindent}{0pt}
\setlength{\parskip}{6pt plus 2pt minus 1pt}
}
\setlength{\emergencystretch}{3em}  % prevent overfull lines
\providecommand{\tightlist}{%
  \setlength{\itemsep}{0pt}\setlength{\parskip}{0pt}}
\setcounter{secnumdepth}{0}
% Redefines (sub)paragraphs to behave more like sections
\ifx\paragraph\undefined\else
\let\oldparagraph\paragraph
\renewcommand{\paragraph}[1]{\oldparagraph{#1}\mbox{}}
\fi
\ifx\subparagraph\undefined\else
\let\oldsubparagraph\subparagraph
\renewcommand{\subparagraph}[1]{\oldsubparagraph{#1}\mbox{}}
\fi

%%% Use protect on footnotes to avoid problems with footnotes in titles
\let\rmarkdownfootnote\footnote%
\def\footnote{\protect\rmarkdownfootnote}

%%% Change title format to be more compact
\usepackage{titling}

% Create subtitle command for use in maketitle
\newcommand{\subtitle}[1]{
  \posttitle{
    \begin{center}\large#1\end{center}
    }
}

\setlength{\droptitle}{-2em}

  \title{I Didn't Sign Up for This: Repression and the Fragmentation of Regime
Forces}
    \pretitle{\vspace{\droptitle}\centering\huge}
  \posttitle{\par}
    \author{David Bowden}
    \preauthor{\centering\large\emph}
  \postauthor{\par}
      \predate{\centering\large\emph}
  \postdate{\par}
    \date{August 25, 2018}

\usepackage{setspace}
\usepackage{longtable}

\usepackage{float}
\let\origtable\table
\let\endorigtable\endtable
\renewenvironment{table}[1][2] {
    \singlespacing
    \expandafter\origtable\expandafter[H]
} {
    \endorigtable
}

\raggedbottom

\usepackage{amsthm}
\newtheorem{theorem}{Theorem}[section]
\newtheorem{lemma}{Lemma}[section]
\theoremstyle{definition}
\newtheorem{definition}{Definition}[section]
\newtheorem{corollary}{Corollary}[section]
\newtheorem{proposition}{Proposition}[section]
\theoremstyle{definition}
\newtheorem{example}{Example}[section]
\theoremstyle{definition}
\newtheorem{exercise}{Exercise}[section]
\theoremstyle{remark}
\newtheorem*{remark}{Remark}
\newtheorem*{solution}{Solution}
\begin{document}
\maketitle
\begin{abstract}
Previous work has shown that the possibility of defection from regime
security forces may deter the use of repression. Yet, such defection
nevertheless occurs in cases such as Syria (The Free Syrian Army),
Libya, and the Democratic Republic of the Congo (M23).
\end{abstract}

\doublespacing

\setlength{\parindent}{1cm}

\hypertarget{introduction}{%
\section{Introduction}\label{introduction}}

When the Arab Spring protests spread to Syria in March of 2011, the
regime of President Bashar al-Assad quickly responded with forceful
repression, including the torture and killing of a 13-year-old boy
(Macleod and Flamand \protect\hyperlink{ref-Macleod2011}{2011}). While
this brutality was presumably intended to deter threats to the regime,
it arguably backfired by provoking the defection of a substantial
portion of the regime military including Colonel Hussein Harmoush, who
expressed a feeling of complicity in the government actions, saying that
``I defected from the Syrian Arab army and took responsibility for
protecting civilians\ldots{} I feel like I am responsible for the deaths
of every single martyr in Syria,'' (Abouzeid
\protect\hyperlink{ref-Abouzeid2011}{2011}). Riyad al-Assad, who
defected from the Syrian Air to form the Free Syrian Army rebel group,
similarly declared his intent to protect protestors and resist the
regime military (Lister \protect\hyperlink{ref-Lister2016}{2016}).
Several other contemporary rebel groups have similar origins, including
M23 in the Democratic Republic of the Congo, and the National Liberation
Army in Libya, suggesting that the phenomenon could be widespread. This
paper thus seeks to answer two question. First, does the use of
repression place regimes at greater risk of desertion and coups? Second,
under what conditions is repression most likely to produce such
outcomes?

The fragmentation of the regime military is a key dynamic in many
conflicts, and potentially explains why the Arab Spring led to civil war
Syria and Libya, but not in other countries. Similar processes have
occurred in numerous other cases, as original data presented herein
shows that more than 15\% of rebel groups since World War II have traced
their origins to the regime military, and an additional 9\% were founded
by civilian regime officials. These rebellions have received little
attention from scholars, and while some existing theories of civil war
onset such as greed theory (Collier and Hoeffler
\protect\hyperlink{ref-Collier2004}{2004}) could potentially account for
them, many, including theories focusing on ethnic discrimination (e.g.
Cederman, Wimmer, and Min \protect\hyperlink{ref-Cederman2010}{2010})
and protest escalation (e.g. Pierskalla
\protect\hyperlink{ref-Pierskalla2010}{2010}), assume that rebellions
originate outside the government. Studying these cases thus offers the
possibility of enhancing our understanding of civil war onset.

Another prominent form of regime fragmentation --- coups d'etat --- has
been the subject of much scholarship. While most of the existing
literature focuses on broader structural conditions affecting coup risk,
some identify a connection between protests and coup occurrence (Casper
and Tyson \protect\hyperlink{ref-Casper2014}{2014}; Johnson and Thyne
\protect\hyperlink{ref-Johnson2018}{2018}). Hendrix and Salehyan
(\protect\hyperlink{ref-Hendrix2017}{2017}) consider the government's
response to protests, finding that the possibility of regime
fragmentation often deters the use of repression. Yet, fragmentation
does occur, suggesting the need for further research analyzing
fragmentation as a dependent variable. Additionally, deterrence effects
create the possibility of endogeneity, but existing studies have not
fully corrected for this concern. This study advances the literature on
coups and repression by making coups a dependent variable, by accounting
for the possibility of endogeneity through the use of an instrumental
variable, and by examining a wider set of cases than previous studies.

This research also contributes to the literature on human rights.
Understandably, most work in this area has focused on the causes of
human rights violations. Several scholars, however, have turned their
attention to the consequences of human rights violations for outcomes
such as foreign direct investment (Blanton and Blanton
\protect\hyperlink{ref-Blanton2007}{2007}) and foreign aid (Lebovic and
Voeten \protect\hyperlink{ref-Lebovic2009}{2009}), and others have
suggested that repression could provoke infighting amongst regime
factions (Hendrix and Salehyan
\protect\hyperlink{ref-Hendrix2017}{2017}). These consequences of human
rights violations could offer insight to preventing abuses in the
future. For example, if regime fragmentation has the potential to
constrain abusive behavior (Hendrix and Salehyan
\protect\hyperlink{ref-Hendrix2017}{2017}), disrupting the flow of
private benefits to soldiers might undermine solidarity and strengthen
this effect. By comparing specific mechanisms linking repression to
regime fragmentation, this study offers the prospect of such policy
recommendations.

I proceed with a review of the literature on regime fragmentation,
including coups, rebellions, and desertion. Next, I articulate three
theoretical processes that could link repression to regime
fragmentation. I then specify a research design to test these
propositions, and present results from fixed-effects and instrumental
variables regression models. I conclude by situating the results in the
broader literature, and by offering suggestions for future research.

\hypertarget{prior-work-on-regime-fragmentation}{%
\section{Prior Work on Regime
Fragmentation}\label{prior-work-on-regime-fragmentation}}

While some forms of political violence are often considered jointly,
such as protests and civil war or civil and international war, coups
d'etat are typically treated as distinct phenomenon, and are not
considered jointly with other forms of violence.\footnote{One notable
  exception is Roessler (\protect\hyperlink{ref-Roessler2011}{2011}).}
Given examples such as the Free Syrian Army, however, I argue that there
is considerable overlap between coups and civil wars. Thus, I analyze
coups and rebellions that originate from the regime jointly under the
umbrella term ``regime fragmentation.''

\hypertarget{dissent-repression-and-coups}{%
\subsection{Dissent, Repression, and
Coups}\label{dissent-repression-and-coups}}

The literature often conceptualizes coup attempts as coordination
problems among elites (Weeks \protect\hyperlink{ref-Weeks2008}{2008};
Svolik
\protect\hyperlink{ref-Svolik2012d}{2012}\protect\hyperlink{ref-Svolik2012d}{b};
Powell \protect\hyperlink{ref-Powell2012a}{2012}). Protests can
potentially solve this coordination problem by revealing the regime's
ability to deter challenges (Casper and Tyson
\protect\hyperlink{ref-Casper2014}{2014}; Johnson and Thyne
\protect\hyperlink{ref-Johnson2018}{2018}). These signals are likely to
be especially influential when protests are non-violent (Johnson and
Thyne \protect\hyperlink{ref-Johnson2018}{2018}), occur in or near the
national capital (Johnson and Thyne
\protect\hyperlink{ref-Johnson2018}{2018}), and are amplified by a free
press (Casper and Tyson \protect\hyperlink{ref-Casper2014}{2014}).
Protests can also serve as a motive for coup attempts by signaling the
illegitimacy of the regime (Johnson and Thyne
\protect\hyperlink{ref-Johnson2018}{2018}), and enhance opportunities
for successful coups by empowering the military (Svolik
\protect\hyperlink{ref-Svolik2012e}{2012}\protect\hyperlink{ref-Svolik2012e}{a}).

While protests are associated with a statistically-significant increase
in the probability of coups (Casper and Tyson
\protect\hyperlink{ref-Casper2014}{2014}; Johnson and Thyne
\protect\hyperlink{ref-Johnson2018}{2018}), responding to protests with
force is not necessarily a wise choice for regimes. Hendrix and Salehyan
(\protect\hyperlink{ref-Hendrix2017}{2017}) argue that the use of
repression can cause backlash within the military, and show that this
possibility deters repressive tactics, particularly when coup-risk is
especially, as is the case for militaries that have previously
experienced infighting, and for protests which emphasize ethnic or
religious identities. While this deterrent effect is substantial, a
great deal of repression occurs nonetheless. Neither Hendrix and
Salehyan (\protect\hyperlink{ref-Hendrix2017}{2017}) nor any other study
to my knowledge examines the reverse relationship assessing, in effect,
whether the assumption that repression leads to regime fragmentation is
correct. Furthermore, the use of repression is likely endogenous to
potential reactions to its use (Ritter and Conrad
\protect\hyperlink{ref-Ritter2016}{2016}). The body of evidence on the
relationship between repression and regime fragmentation would therefore
be strengthened by analyses using causal identification techniques such
as the instrumental variable analysis presented here.

\hypertarget{repression-and-civil-war}{%
\subsection{Repression and Civil War}\label{repression-and-civil-war}}

Many scholars of political violence and human rights have examined the
``repression-dissent nexus,'' often focusing on the role of government
repression in escalating unrest. There is widespread agreement that
repression can backfire and escalate dissident activities, though the
conditions under which this occurs are contested. Early work in this
area disaggregated dissident activity, showing that repression reduces
non-violent dissent while increasing violent opposition (Lichbach
\protect\hyperlink{ref-Lichbach1987}{1987}; Moore
\protect\hyperlink{ref-Moore1998}{1998}). Rasler
(\protect\hyperlink{ref-Rasler1996}{1996}) emphasizes temporal dynamics,
providing evidence that repression reduces dissident activity in the
short-run while increasing it in the long-run. Pierskalla
(\protect\hyperlink{ref-Pierskalla2010}{2010}) criticizes prior studies
for their lack of attention to strategic interplay, and finds that
escalation should only occur in the presence of a third-party threat.
The preceding findings concern primarily the qualitative character of
dissent, rather than the quantitative volume. The consistency of
repressive or accommodative policies (Lichbach
\protect\hyperlink{ref-Lichbach1987}{1987}), the forcefulness of
repressive tactics (Hegre et al.
\protect\hyperlink{ref-Hegre2001}{2001}; Pierskalla
\protect\hyperlink{ref-Pierskalla2010}{2010}), and prior history of
civil conflict (Bell and Murdie \protect\hyperlink{ref-Bell2018}{2018})
have been posited as explanations for the aggregate level of dissent.

The preceding studies generally define escalation in relative terms,
leaving unclear the frequency with which the violence they observe
aligns with scholarly definitions of civil war.\footnote{Such
  definitions typically entail a minimum threshold of severity such as
  25 fatalities per calendar year, a substantial degree of organization
  on both sides, and some amount of competitiveness between the two
  sides. For a representative example, see Pettersson and Eck
  (\protect\hyperlink{ref-Pettersson2018}{2018}).} However, several
works do focus specifically on civil war as a form of escalation. Some
scholars conceptualize repression and civil war as distinct equilibria
(Besley and Persson \protect\hyperlink{ref-Besley2009}{2009}; Choi and
Kim \protect\hyperlink{ref-Choi2018}{2018}). The probability that
repression is met by effective resistance (the civil war outcome) rather
than remaining one-sided is shown to increase with the material value of
winning control of the government (Besley and Persson
\protect\hyperlink{ref-Besley2009}{2009}), decrease with the
inclusiveness of political institutions (Besley and Persson
\protect\hyperlink{ref-Besley2009}{2009}; Choi and Kim
\protect\hyperlink{ref-Choi2018}{2018}), and increase with the size of
the dominant ethnic group relative to the size of the minimum winning
coalition (Choi and Kim \protect\hyperlink{ref-Choi2018}{2018}). Others
treat repression and civil war as sequential steps in a process of
escalation. In this view, repression often proves counterproductive as
it decreases support for the regime (Young
\protect\hyperlink{ref-Young2013}{2013}) and increases the probability
of civil war. Interestingly, this relationship does not appear to be
conditional on the efficacy of repression, as effective repression
increases the risk of civil war by further inflaming tensions, while
ineffective repression does so by emboldening dissidents (Davenport,
Armstrong II, and Lichbach, \protect\hyperlink{ref-Davenportnd}{n.d.}).
In addition to being associated with the onset of civil conflict,
repression predicts increased violence by existing rebel groups
(Shellman, Levey, and Young \protect\hyperlink{ref-Shellman2013}{2013}).

The studies reviewed here have consistently found a link between
repression and the onset of civil conflict. However, tests of specific
theoretical mechanisms have been less conclusive. A potential reason for
this is that while all of the theories above assume that rebel groups
emerge from non-state dissident movements, original data presented here
reveals that a large number of rebellions included in commonly-used
datasets are launched by members of the regime. Differences in the
conditions under which repression produces these two types of rebellions
could account for the inconclusive results on mechanisms observed thus
far. This analysis could shed light on such a possibility.

\hypertarget{theorizing-regime-fragmentation}{%
\section{Theorizing Regime
Fragmentation}\label{theorizing-regime-fragmentation}}

\emph{H1a: Repression increases the probability of coups} \emph{H1b:
Repression increases the probability of regime rebellion}

\emph{H2: There is a positive interaction between the level of
repression and the share of the military with ethnic ties to the
repressed}

\emph{H3: There is a positive interaction between the level of
repression and the level of infighting amongst military factions}

\hypertarget{research-design}{%
\section{Research Design}\label{research-design}}

\hypertarget{control-variables}{%
\subsection{Control Variables}\label{control-variables}}

As coups have been shown to be most prevalent in relatively poor states
(Londregan and Poole \protect\hyperlink{ref-Londregan1990}{1990}), I
include the per capita gross domestic product using data from Gleditsch
(\protect\hyperlink{ref-Gleditsch2002a}{2002}) (version 6.0 beta) for
the period 1950-2011.

\hypertarget{results}{%
\section{Results}\label{results}}

\singlespacing

\begin{table}[htbp]\centering
\def\sym#1{\ifmmode^{#1}\else\(^{#1}\)\fi}
\caption{\label{tab1} Fixed-Effects Logit Models of Coups}
\begin{tabular}{l*{4}{c}}
\hline\hline
                    &\multicolumn{1}{c}{(1)}         &\multicolumn{1}{c}{(2)}         &\multicolumn{1}{c}{(3)}         &\multicolumn{1}{c}{(4)}         \\
                    &        Coup         &Regime Rebellion         &        Coup         &Regime Rebellion         \\
\hline
Latent Protection Score&       -0.50\sym{***}&       -0.94\sym{***}&                     &                     \\
                    &      (0.11)         &      (0.23)         &                     &                     \\
NAVCO Repression    &                     &                     &        0.16\sym{*}  &        0.25         \\
                    &                     &                     &      (0.07)         &      (0.14)         \\
Autocracy           &       -0.65\sym{***}&       -0.85\sym{**} &       -0.52\sym{***}&       -0.58         \\
                    &      (0.15)         &      (0.33)         &      (0.15)         &      (0.32)         \\
Democracy           &       -0.24         &       -0.19         &       -0.31         &       -0.37         \\
                    &      (0.20)         &      (0.48)         &      (0.20)         &      (0.48)         \\
Military Regime     &        0.33         &        0.17         &        0.37\sym{*}  &        0.34         \\
                    &      (0.18)         &      (0.46)         &      (0.18)         &      (0.45)         \\
log GDPpc           &       -0.24         &        0.08         &       -0.30         &       -0.05         \\
                    &      (0.17)         &      (0.40)         &      (0.17)         &      (0.38)         \\
log Population      &       -1.54\sym{***}&       -1.01\sym{*}  &       -1.35\sym{***}&       -0.58         \\
                    &      (0.18)         &      (0.40)         &      (0.17)         &      (0.36)         \\
Civil Conflict      &        0.22         &       -0.98\sym{*}  &        0.32         &       -0.67         \\
                    &      (0.19)         &      (0.42)         &      (0.20)         &      (0.45)         \\
\hline
N                   &        4340         &        2210         &        4340         &        2210         \\
\hline\hline
\multicolumn{5}{l}{\footnotesize \sym{*} \(p<0.05\), \sym{**} \(p<0.01\), \sym{***} \(p<0.001\)}\\
\end{tabular}
\end{table}


\doublespacing

\singlespacing

\begin{table}[htbp]\centering
\def\sym#1{\ifmmode^{#1}\else\(^{#1}\)\fi}
\caption{\label{tab1} Fixed-Effects Logit Models of Coups}
\begin{tabular}{l*{2}{c}}
\hline\hline
                    &\multicolumn{1}{c}{(1)}         &\multicolumn{1}{c}{(2)}         \\
                    &        Coup         &        Coup         \\
\hline
Latent Protection Score&       -0.60\sym{*}  &       -0.17         \\
                    &      (0.29)         &      (0.36)         \\
Latent Protection Score&        0.00         &                     \\
                    &         (.)         &                     \\
% Officers Discriminated&       -0.01         &       -0.04\sym{*}  \\
                    &      (0.01)         &      (0.02)         \\
% Officers Discriminated&        0.00         &                     \\
                    &         (.)         &                     \\
Autocracy           &       -0.47         &       -0.58         \\
                    &      (0.39)         &      (0.40)         \\
Democracy           &       -0.23         &       -0.27         \\
                    &      (0.62)         &      (0.61)         \\
Military Regime     &       -0.75         &       -0.79         \\
                    &      (0.57)         &      (0.57)         \\
log GDPpc           &       -0.09         &       -0.01         \\
                    &      (0.39)         &      (0.40)         \\
log Population      &       -1.69\sym{***}&       -1.78\sym{***}\\
                    &      (0.43)         &      (0.46)         \\
Civil Conflict      &       -0.11         &       -0.27         \\
                    &      (0.43)         &      (0.44)         \\
Latent Protection Score $\times$ % Officers Discriminated&                     &       -0.02         \\
                    &                     &      (0.01)         \\
\hline
N                   &         778         &         778         \\
\hline\hline
\multicolumn{3}{l}{\footnotesize \sym{*} \(p<0.05\), \sym{**} \(p<0.01\), \sym{***} \(p<0.001\)}\\
\end{tabular}
\end{table}


\doublespacing

\hypertarget{references}{%
\section*{References}\label{references}}
\addcontentsline{toc}{section}{References}

\markboth{REFERENCES}{}

\indent

\setlength{\parindent}{-0.2in}
\setlength{\leftskip}{0.2in}
\setlength{\parskip}{8pt}

\singlespacing

\hypertarget{refs}{}
\leavevmode\hypertarget{ref-Abouzeid2011}{}%
Abouzeid, Rania. 2011. ``The Soldier Who Gave up on Assad to Protect
Syria's People.'' \emph{Time}.
https://content.time.com/time/world/article/0,8599,2077348,00.html.

\leavevmode\hypertarget{ref-Bell2018}{}%
Bell, Sam R, and Amanda Murdie. 2018. ``The Apparatus for Violence:
Repression, Violent Protest, and Civil War in a Cross-National
Framework.'' \emph{Conflict Management and Peace Science} 35 (4):
336--54.

\leavevmode\hypertarget{ref-Besley2009}{}%
Besley, Timothy, and Torsten Persson. 2009. ``Repression or Civil War?''
\emph{American Economic Review} 99 (2): 292--97.

\leavevmode\hypertarget{ref-Blanton2007}{}%
Blanton, Shannon Lindsey, and Robert G. Blanton. 2007. ``What Attracts
Foreign Investors? An Examination of Human Rights and Foreign Direct
Investment.'' \emph{The Journal of Politics} 69 (1): 143--55.

\leavevmode\hypertarget{ref-Casper2014}{}%
Casper, Brett Allen, and Scott A. Tyson. 2014. ``Popular Protest and
Elite Coordination in a Coup d'état.'' \emph{Journal of Politics} 76
(2): 548--64.

\leavevmode\hypertarget{ref-Cederman2010}{}%
Cederman, Lars-Erik, Andreas Wimmer, and Brian Min. 2010. ``Why Do
Ethnic Groups Rebel?: New Data and Analysis.'' \emph{World Politics} 62
(1): 87--98.

\leavevmode\hypertarget{ref-Choi2018}{}%
Choi, Hyun Jin, and Dongsuk Kim. 2018. ``Coup, Riot, War: How Political
Institutions and Ethnic Politics Shape Alternative Forms of Political
Violence.'' \emph{Terrorism and Political Violence} 30 (4): 718--39.

\leavevmode\hypertarget{ref-Collier2004}{}%
Collier, Paul, and Anke Hoeffler. 2004. ``Greed and Grievance in Civil
War.'' \emph{Oxford Economic Papers} 56 (4): 563--95.

\leavevmode\hypertarget{ref-Davenportnd}{}%
Davenport, Christian, David A. Armstrong II, and Mark I. Lichbach. n.d.
``From Mountains to Movements : Dissent , Repression and Escalation to
Civil War.'' \emph{Unpublished}.

\leavevmode\hypertarget{ref-Gleditsch2002a}{}%
Gleditsch, Kristian Skrede. 2002. ``Expanded Trade and GDP Data.''
\emph{Journal of Conflict Resolution} 46 (5): 712--24.

\leavevmode\hypertarget{ref-Hegre2001}{}%
Hegre, Havard, Tanja Ellison, Scott Gates, and Nils Petter Gleditsch.
2001. ``Toward a Democratic Civil Peace? Democracy, Political Change,
and Civil War, 1816-1992.'' \emph{American Political Science Review} 95
(1): 33--48.

\leavevmode\hypertarget{ref-Hendrix2017}{}%
Hendrix, Cullen S., and Idean Salehyan. 2017. ``A House Divided: Threat
Perception, Military Factionalism, and Repression in Africa.''
\emph{Journal of Conflict Resolution} 61 (8): 1653--81.

\leavevmode\hypertarget{ref-Johnson2018}{}%
Johnson, Jaclyn, and Clayton L. Thyne. 2018. ``Squeaky Wheels and Troop
Loyalty: How Domestic Protests Influence Coups d'état, 19512005.''
\emph{Journal of Conflict Resolution} 62 (3): 597--625.

\leavevmode\hypertarget{ref-Lebovic2009}{}%
Lebovic, James H., and Erik Voeten. 2009. ``The Cost of Shame:
International Organizations and Foreign Aid in the Punishing of Human
Rights Violators.'' \emph{Journal of Peace Research} 46 (1): 79--97.

\leavevmode\hypertarget{ref-Lichbach1987}{}%
Lichbach, Mark Irving. 1987. ``Deterrence or Escalation? The Puzzle of
Aggregate Studies of Repression and Dissent.'' \emph{Journal of Conflict
Resolution} 31 (2): 266--97.

\leavevmode\hypertarget{ref-Lister2016}{}%
Lister, Charles. 2016. ``The Free Syrian Army: A Decentralized Insurgent
Brand.'' \emph{Brookings Project on U.S. Relations with the Islamic
World}, no. 26.

\leavevmode\hypertarget{ref-Londregan1990}{}%
Londregan, John B., and Keith T. Poole. 1990. ``Poverty, the Coup Trap,
and the Seizure of Executive Power.'' \emph{World Politics} 42 (2):
151--83.

\leavevmode\hypertarget{ref-Macleod2011}{}%
Macleod, Hugh, and Annasofie Flamand. 2011. ``Tortured and Killed: Hamza
Al-Khateeb, Age 13.'' \emph{Al Jazeera}.
https://www.aljazeera.com/indepth/features/2011/05/201153185927813389.html.

\leavevmode\hypertarget{ref-Moore1998}{}%
Moore, Will H. 1998. ``Repression and Dissent: Substitution, Context,
and Timing.'' \emph{American Journal of Political Science} 42 (3):
851--73.

\leavevmode\hypertarget{ref-Pettersson2018}{}%
Pettersson, Therése, and Kristine Eck. 2018. ``Organized Violence,
19892017.'' \emph{Journal of Peace Research} 55 (4): 535--47.

\leavevmode\hypertarget{ref-Pierskalla2010}{}%
Pierskalla, Jan Henryk. 2010. ``Protest, Deterrence, and Escalation: The
Strategic Calculus of Government Repression.'' \emph{Journal of Conflict
Resolution} 54 (1): 117--45.

\leavevmode\hypertarget{ref-Powell2012a}{}%
Powell, Jonathan. 2012. ``Determinants of the Attempting and Outcome of
Coups d'état.'' \emph{Journal of Conflict Resolution} 56 (6): 1017--40.

\leavevmode\hypertarget{ref-Rasler1996}{}%
Rasler, Karen. 1996. ``Concessions, Repression, and Political Protest in
the Iranian Revolution.'' \emph{American Sociological Review} 61 (1):
132.

\leavevmode\hypertarget{ref-Ritter2016}{}%
Ritter, Emily Hencken, and Courtenay R. Conrad. 2016. ``Preventing and
Responding to Dissent: The Observational Challenges of Explaining
Strategic Repression.'' \emph{American Political Science Review} 110
(01): 85--99.

\leavevmode\hypertarget{ref-Roessler2011}{}%
Roessler, Philip. 2011. ``The Enemy Within: Personal Rule, Coups, and
Civil War in Africa.'' \emph{World Politics} 63 (2): 300--346.

\leavevmode\hypertarget{ref-Shellman2013}{}%
Shellman, Stephen M., Brian P. Levey, and Joseph K. Young. 2013.
``Shifting Sands: Explaining and Predicting Phase Shifts by Dissident
Organizations.'' \emph{Journal of Peace Research} 50 (3): 319--36.

\leavevmode\hypertarget{ref-Svolik2012e}{}%
Svolik, Milan W. 2012a. ``Contracting on Violence: The Moral Hazard in
Authoritarian Repression and Military Intervention in Politics.''
\emph{Journal of Conflict Resolution} 57 (5): 765--94.

\leavevmode\hypertarget{ref-Svolik2012d}{}%
---------. 2012b. \emph{The Politics of Authoritarian Rule}. Cambridge:
Cambridge University Press.

\leavevmode\hypertarget{ref-Weeks2008}{}%
Weeks, Jessica L. 2008. ``Autocratic Audience Costs: Regime Type and
Signaling Resolve.'' \emph{International Organization}.
/core/journals/international-organization/article/autocratic-audience-costs-regime-type-and-signaling-resolve/3EB4447D7E4584B523BFA6D5AC1542D3.

\leavevmode\hypertarget{ref-Young2013}{}%
Young, Joseph K. 2013. ``Repression, Dissent, and the Onset of Civil
War.'' \emph{Political Research Quarterly} 66 (3): 516--32.


\end{document}
