\documentclass[12pt,]{article}
\usepackage[]{tgpagella}
\usepackage{amssymb,amsmath}
\usepackage{ifxetex,ifluatex}
\usepackage{fixltx2e} % provides \textsubscript
\ifnum 0\ifxetex 1\fi\ifluatex 1\fi=0 % if pdftex
  \usepackage[T1]{fontenc}
  \usepackage[utf8]{inputenc}
\else % if luatex or xelatex
  \ifxetex
    \usepackage{mathspec}
  \else
    \usepackage{fontspec}
  \fi
  \defaultfontfeatures{Ligatures=TeX,Scale=MatchLowercase}
\fi
% use upquote if available, for straight quotes in verbatim environments
\IfFileExists{upquote.sty}{\usepackage{upquote}}{}
% use microtype if available
\IfFileExists{microtype.sty}{%
\usepackage{microtype}
\UseMicrotypeSet[protrusion]{basicmath} % disable protrusion for tt fonts
}{}
\usepackage[margin=1in]{geometry}
\usepackage{hyperref}
\hypersetup{unicode=true,
            pdftitle={Repression as a Cause of Coups},
            pdfauthor={David Bowden},
            pdfborder={0 0 0},
            breaklinks=true}
\urlstyle{same}  % don't use monospace font for urls
\usepackage{longtable,booktabs}
\usepackage{graphicx,grffile}
\makeatletter
\def\maxwidth{\ifdim\Gin@nat@width>\linewidth\linewidth\else\Gin@nat@width\fi}
\def\maxheight{\ifdim\Gin@nat@height>\textheight\textheight\else\Gin@nat@height\fi}
\makeatother
% Scale images if necessary, so that they will not overflow the page
% margins by default, and it is still possible to overwrite the defaults
% using explicit options in \includegraphics[width, height, ...]{}
\setkeys{Gin}{width=\maxwidth,height=\maxheight,keepaspectratio}
\IfFileExists{parskip.sty}{%
\usepackage{parskip}
}{% else
\setlength{\parindent}{0pt}
\setlength{\parskip}{6pt plus 2pt minus 1pt}
}
\setlength{\emergencystretch}{3em}  % prevent overfull lines
\providecommand{\tightlist}{%
  \setlength{\itemsep}{0pt}\setlength{\parskip}{0pt}}
\setcounter{secnumdepth}{0}
% Redefines (sub)paragraphs to behave more like sections
\ifx\paragraph\undefined\else
\let\oldparagraph\paragraph
\renewcommand{\paragraph}[1]{\oldparagraph{#1}\mbox{}}
\fi
\ifx\subparagraph\undefined\else
\let\oldsubparagraph\subparagraph
\renewcommand{\subparagraph}[1]{\oldsubparagraph{#1}\mbox{}}
\fi

%%% Use protect on footnotes to avoid problems with footnotes in titles
\let\rmarkdownfootnote\footnote%
\def\footnote{\protect\rmarkdownfootnote}

%%% Change title format to be more compact
\usepackage{titling}

% Create subtitle command for use in maketitle
\newcommand{\subtitle}[1]{
  \posttitle{
    \begin{center}\large#1\end{center}
    }
}

\setlength{\droptitle}{-2em}

  \title{Repression as a Cause of Coups}
    \pretitle{\vspace{\droptitle}\centering\huge}
  \posttitle{\par}
    \author{David Bowden}
    \preauthor{\centering\large\emph}
  \postauthor{\par}
      \predate{\centering\large\emph}
  \postdate{\par}
    \date{August 24, 2018}

\usepackage{setspace}
\usepackage{longtable}

\usepackage{float}
\let\origtable\table
\let\endorigtable\endtable
\renewenvironment{table}[1][2] {
    \singlespacing
    \expandafter\origtable\expandafter[H]
} {
    \endorigtable
}

\raggedbottom

% \setlength{\parskip}{1cm plus4mm minus3mm}

\usepackage{amsthm}
\newtheorem{theorem}{Theorem}[section]
\newtheorem{lemma}{Lemma}[section]
\theoremstyle{definition}
\newtheorem{definition}{Definition}[section]
\newtheorem{corollary}{Corollary}[section]
\newtheorem{proposition}{Proposition}[section]
\theoremstyle{definition}
\newtheorem{example}{Example}[section]
\theoremstyle{definition}
\newtheorem{exercise}{Exercise}[section]
\theoremstyle{remark}
\newtheorem*{remark}{Remark}
\newtheorem*{solution}{Solution}
\begin{document}
\maketitle
\begin{abstract}
Previous work has shown that the possibility of defection from regime
security forces may deter the use of repression. Yet, such defection
nevertheless occurs in cases such as Syria (The Free Syrian Army),
Libya, and the Democratic Republic of the Congo (M23).
\end{abstract}

\doublespacing

\hypertarget{introduction}{%
\section{Introduction}\label{introduction}}

When the Arab Spring protests spread to Syria in March of 2011, the
regime of President Bashar al-Assad quickly responded with forceful
repression, including the torture and killing of a 13-year-old boy
(Macleod and Flamand \protect\hyperlink{ref-Macleod2011}{2011}). While
this brutality was presumably intended to deter threats to the regime,
it arguably backfired by provoking the defection of a substantial
portion of the regime military including Colonel Hussein Harmoush, who
expressed a feeling of complicity in the government actions, saying that
``I defected from the Syrian Arab army and took responsibility for
protecting civilians\ldots{} I feel like I am responsible for the deaths
of every single martyr in Syria,'' (Abouzeid
\protect\hyperlink{ref-Abouzeid2011}{2011}). Riyad al-Assad, who
defected from the Syrian Air to form the Free Syrian Army rebel group,
similarly declared his intent to protect protestors and resist the
regime military (Lister \protect\hyperlink{ref-Lister2016}{2016}).
Several other contemporary rebel groups have similar origins, including
M23 in the Democratic Republic of the Congo, and the National Liberation
Army in Libya, suggesting that the phenomenon could be widespread. This
paper thus seeks to answer two question. First, does the use of
repression place regimes at greater risk of desertion and coups? Second,
under what conditions is repression most likely to produce such
outcomes?

The fragmentation of the regime military is a key dynamic in many
conflicts, and potentially explains why the Arab Spring led to civil war
Syria and Libya, but not in other countries. Similar processes have
occurred in numerous other cases, as more than 15\% of rebel groups
since World War II have traced their origins to the regime military, and
an additional 9\% were founded by civilian regime officials (Bowden
\protect\hyperlink{ref-Bowden2017}{2017}). These rebellions have
received little attention from scholars, and while some existing
theories of civil war onset such as greed theory (Collier and Hoeffler
\protect\hyperlink{ref-Collier2004}{2004}) could potentially account for
them, many, including theories focusing on ethnic discrimination (e.g.
Cederman, Wimmer, and Min \protect\hyperlink{ref-Cederman2010}{2010})
and protest escalation (e.g. Pierskalla
\protect\hyperlink{ref-Pierskalla2010}{2010}), assume that rebellions
originate outside the government. Studying these cases thus offers the
possibility of enhancing our understanding of civil war onset.

Another prominent form of regime fragmentation --- coups d'etat --- has
been the subject of much scholarship. While most of the existing
literature focuses on broader structural conditions affecting coup risk,
some identify a connection between protests and coup occurrence (Casper
and Tyson \protect\hyperlink{ref-Casper2014}{2014}; Johnson and Thyne
\protect\hyperlink{ref-Johnson2018}{2018}). Hendrix and Salehyan
(\protect\hyperlink{ref-Hendrix2017}{2017}) consider the government's
response to protests, finding that the possibility of regime
fragmentation often deters the use of repression. Yet, fragmentation
does occur, suggesting the need for further research analyzing
fragmentation as a dependent variable. Additionally, deterrence effects
create the possibility of endogeneity, but existing studies have not
fully corrected for this concern. This study advances the literature on
coups and repression by making coups a dependent variable, by accounting
for the possibility of endogeneity through the use of an instrumental
variable, and by examining a wider set of cases than previous studies.

This research also contributes to the literature on human rights.
Understandably, most work in this area has focused on the causes of
human rights violations. Several scholars, however, have turned their
attention to the consequences of human rights violations for outcomes
such as foreign direct investment (Blanton and Blanton
\protect\hyperlink{ref-Blanton2007}{2007}) and foreign aid (Lebovic and
Voeten \protect\hyperlink{ref-Lebovic2009}{2009}), and others have
suggested that repression could provoke infighting amongst regime
factions (Hendrix and Salehyan
\protect\hyperlink{ref-Hendrix2017}{2017}). These consequences of human
rights violations could offer insight to preventing abuses in the
future. For example, if regime fragmentation has the potential to
constrain abusive behavior (Hendrix and Salehyan
\protect\hyperlink{ref-Hendrix2017}{2017}), disrupting the flow of
private benefits to soldiers might undermine solidarity and strengthen
this effect. By comparing specific mechanisms linking repression to
regime fragmentation, this study offers the prospect of such policy
recommendations.

I proceed with a review of the literature on regime fragmentation,
including coups, rebellions, and desertion. Next, I articulate three
theoretical processes that could link repression to regime
fragmentation. I then specify a research design to test these
propositions, and present results from fixed-effects and instrumental
variables regression models. I conclude by situating the results in the
broader literature, and by offering suggestions for future research.

\hypertarget{prior-work-on-regime-fragmentation}{%
\section{Prior Work on Regime
Fragmentation}\label{prior-work-on-regime-fragmentation}}

While some forms of political violence are often considered jointly,
such as protests and civil war or civil and international war, coups
d'etat are typically treated as distinct phenomenon, and are not
considered jointly with other forms of violence.\footnote{One notable
  exception is Roessler (\protect\hyperlink{ref-Roessler2011}{2011})}

\hypertarget{research-design}{%
\section{Research Design}\label{research-design}}

\hypertarget{control-variables}{%
\subsection{Control Variables}\label{control-variables}}

As coups have been shown to be most prevalent in relatively poor states
(Londregan and Poole \protect\hyperlink{ref-Londregan1990}{1990}), I
include the per capita gross domestic product using data from Gleditsch
(\protect\hyperlink{ref-Gleditsch2002a}{2002}) (version 6.0 beta) for
the period 1950-2011.

\hypertarget{results}{%
\section{Results}\label{results}}

\singlespacing

\begin{table}[htbp]\centering
\def\sym#1{\ifmmode^{#1}\else\(^{#1}\)\fi}
\caption{\label{tab1} Fixed-Effects Logit Models of Coups}
\begin{tabular}{l*{4}{c}}
\hline\hline
                    &\multicolumn{1}{c}{(1)}         &\multicolumn{1}{c}{(2)}         &\multicolumn{1}{c}{(3)}         &\multicolumn{1}{c}{(4)}         \\
                    &        Coup         &Regime Rebellion         &        Coup         &Regime Rebellion         \\
\hline
Latent Protection Score&       -0.50\sym{***}&       -0.94\sym{***}&                     &                     \\
                    &      (0.11)         &      (0.23)         &                     &                     \\
NAVCO Repression    &                     &                     &        0.16\sym{*}  &        0.25         \\
                    &                     &                     &      (0.07)         &      (0.14)         \\
Autocracy           &       -0.65\sym{***}&       -0.85\sym{**} &       -0.52\sym{***}&       -0.58         \\
                    &      (0.15)         &      (0.33)         &      (0.15)         &      (0.32)         \\
Democracy           &       -0.24         &       -0.19         &       -0.31         &       -0.37         \\
                    &      (0.20)         &      (0.48)         &      (0.20)         &      (0.48)         \\
Military Regime     &        0.33         &        0.17         &        0.37\sym{*}  &        0.34         \\
                    &      (0.18)         &      (0.46)         &      (0.18)         &      (0.45)         \\
log GDPpc           &       -0.24         &        0.08         &       -0.30         &       -0.05         \\
                    &      (0.17)         &      (0.40)         &      (0.17)         &      (0.38)         \\
log Population      &       -1.54\sym{***}&       -1.01\sym{*}  &       -1.35\sym{***}&       -0.58         \\
                    &      (0.18)         &      (0.40)         &      (0.17)         &      (0.36)         \\
Civil Conflict      &        0.22         &       -0.98\sym{*}  &        0.32         &       -0.67         \\
                    &      (0.19)         &      (0.42)         &      (0.20)         &      (0.45)         \\
\hline
N                   &        4340         &        2210         &        4340         &        2210         \\
\hline\hline
\multicolumn{5}{l}{\footnotesize \sym{*} \(p<0.05\), \sym{**} \(p<0.01\), \sym{***} \(p<0.001\)}\\
\end{tabular}
\end{table}


\doublespacing

\hypertarget{references}{%
\section*{References}\label{references}}
\addcontentsline{toc}{section}{References}

\markboth{REFERENCES}{}

\indent

\setlength{\parindent}{-0.2in}
\setlength{\leftskip}{0.2in}
\setlength{\parskip}{8pt}

\singlespacing

\hypertarget{refs}{}
\leavevmode\hypertarget{ref-Abouzeid2011}{}%
Abouzeid, Rania. 2011. ``The Soldier Who Gave up on Assad to Protect
Syria's People.'' \emph{Time}.
https://content.time.com/time/world/article/0,8599,2077348,00.html.

\leavevmode\hypertarget{ref-Blanton2007}{}%
Blanton, Shannon Lindsey, and Robert G. Blanton. 2007. ``What Attracts
Foreign Investors? An Examination of Human Rights and Foreign Direct
Investment.'' \emph{The Journal of Politics} 69 (1): 143--55.
\url{https://doi.org/10.1111/j.1468-2508.2007.00500.x}.

\leavevmode\hypertarget{ref-Bowden2017}{}%
Bowden, David F. 2017. ``Politics Among Rebels: The Causes of Division
Between Dissidents.'' Ph.D. Diss., University of Illinois,
Urbana-Champaign.

\leavevmode\hypertarget{ref-Casper2014}{}%
Casper, Brett Allen, and Scott A. Tyson. 2014. ``Popular Protest and
Elite Coordination in a Coup d'état.'' \emph{Journal of Politics} 76
(2): 548--64.

\leavevmode\hypertarget{ref-Cederman2010}{}%
Cederman, Lars-Erik, Andreas Wimmer, and Brian Min. 2010. ``Why Do
Ethnic Groups Rebel?: New Data and Analysis.'' \emph{World Politics} 62
(1): 87--98.

\leavevmode\hypertarget{ref-Collier2004}{}%
Collier, Paul, and Anke Hoeffler. 2004. ``Greed and Grievance in Civil
War.'' \emph{Oxford Economic Papers} 56 (4): 563--95.

\leavevmode\hypertarget{ref-Gleditsch2002a}{}%
Gleditsch, Kristian Skrede. 2002. ``Expanded Trade and GDP Data.''
\emph{Journal of Conflict Resolution} 46 (5): 712--24.

\leavevmode\hypertarget{ref-Hendrix2017}{}%
Hendrix, Cullen S., and Idean Salehyan. 2017. ``A House Divided: Threat
Perception, Military Factionalism, and Repression in Africa.''
\emph{Journal of Conflict Resolution} 61 (8): 1653--81.
\url{https://doi.org/10.1177/0022002715620473}.

\leavevmode\hypertarget{ref-Johnson2018}{}%
Johnson, Jaclyn, and Clayton L. Thyne. 2018. ``Squeaky Wheels and Troop
Loyalty: How Domestic Protests Influence Coups d'état, 19512005.''
\emph{Journal of Conflict Resolution} 62 (3): 597--625.
\url{https://doi.org/10.1177/0022002716654742}.

\leavevmode\hypertarget{ref-Lebovic2009}{}%
Lebovic, James H., and Erik Voeten. 2009. ``The Cost of Shame:
International Organizations and Foreign Aid in the Punishing of Human
Rights Violators.'' \emph{Journal of Peace Research} 46 (1): 79--97.
\url{https://doi.org/10.1177/0022343308098405}.

\leavevmode\hypertarget{ref-Lister2016}{}%
Lister, Charles. 2016. ``The Free Syrian Army: A Decentralized Insurgent
Brand.'' \emph{Brookings Project on U.S. Relations with the Islamic
World}, no. 26.

\leavevmode\hypertarget{ref-Londregan1990}{}%
Londregan, John B., and Keith T. Poole. 1990. ``Poverty, the Coup Trap,
and the Seizure of Executive Power.'' \emph{World Politics} 42 (2):
151--83.

\leavevmode\hypertarget{ref-Macleod2011}{}%
Macleod, Hugh, and Annasofie Flamand. 2011. ``Tortured and Killed: Hamza
Al-Khateeb, Age 13.'' \emph{Al Jazeera}.
https://www.aljazeera.com/indepth/features/2011/05/201153185927813389.html.

\leavevmode\hypertarget{ref-Pierskalla2010}{}%
Pierskalla, Jan Henryk. 2010. ``Protest, Deterrence, and Escalation: The
Strategic Calculus of Government Repression.'' \emph{Journal of Conflict
Resolution} 54 (1): 117--45.

\leavevmode\hypertarget{ref-Roessler2011}{}%
Roessler, Philip. 2011. ``The Enemy Within: Personal Rule, Coups, and
Civil War in Africa.'' \emph{World Politics} 63 (2): 300--346.


\end{document}
