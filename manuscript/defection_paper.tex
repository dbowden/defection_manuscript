\documentclass[12pt,]{article}
\usepackage[]{tgpagella}
\usepackage{amssymb,amsmath}
\usepackage{ifxetex,ifluatex}
\usepackage{fixltx2e} % provides \textsubscript
\ifnum 0\ifxetex 1\fi\ifluatex 1\fi=0 % if pdftex
  \usepackage[T1]{fontenc}
  \usepackage[utf8]{inputenc}
\else % if luatex or xelatex
  \ifxetex
    \usepackage{mathspec}
  \else
    \usepackage{fontspec}
  \fi
  \defaultfontfeatures{Ligatures=TeX,Scale=MatchLowercase}
\fi
% use upquote if available, for straight quotes in verbatim environments
\IfFileExists{upquote.sty}{\usepackage{upquote}}{}
% use microtype if available
\IfFileExists{microtype.sty}{%
\usepackage{microtype}
\UseMicrotypeSet[protrusion]{basicmath} % disable protrusion for tt fonts
}{}
\usepackage[margin=1in]{geometry}
\usepackage{hyperref}
\hypersetup{unicode=true,
            pdftitle={I Didn't Sign Up for This: Repression and the Fragmentation of Regime Forces},
            pdfborder={0 0 0},
            breaklinks=true}
\urlstyle{same}  % don't use monospace font for urls
\usepackage{longtable,booktabs}
\usepackage{graphicx}
% grffile has become a legacy package: https://ctan.org/pkg/grffile
\IfFileExists{grffile.sty}{%
\usepackage{grffile}
}{}
\makeatletter
\def\maxwidth{\ifdim\Gin@nat@width>\linewidth\linewidth\else\Gin@nat@width\fi}
\def\maxheight{\ifdim\Gin@nat@height>\textheight\textheight\else\Gin@nat@height\fi}
\makeatother
% Scale images if necessary, so that they will not overflow the page
% margins by default, and it is still possible to overwrite the defaults
% using explicit options in \includegraphics[width, height, ...]{}
\setkeys{Gin}{width=\maxwidth,height=\maxheight,keepaspectratio}
\IfFileExists{parskip.sty}{%
\usepackage{parskip}
}{% else
\setlength{\parindent}{0pt}
\setlength{\parskip}{6pt plus 2pt minus 1pt}
}
\setlength{\emergencystretch}{3em}  % prevent overfull lines
\providecommand{\tightlist}{%
  \setlength{\itemsep}{0pt}\setlength{\parskip}{0pt}}
\setcounter{secnumdepth}{0}
% Redefines (sub)paragraphs to behave more like sections
\ifx\paragraph\undefined\else
\let\oldparagraph\paragraph
\renewcommand{\paragraph}[1]{\oldparagraph{#1}\mbox{}}
\fi
\ifx\subparagraph\undefined\else
\let\oldsubparagraph\subparagraph
\renewcommand{\subparagraph}[1]{\oldsubparagraph{#1}\mbox{}}
\fi

%%% Use protect on footnotes to avoid problems with footnotes in titles
\let\rmarkdownfootnote\footnote%
\def\footnote{\protect\rmarkdownfootnote}

%%% Change title format to be more compact
\usepackage{titling}

% Create subtitle command for use in maketitle
\providecommand{\subtitle}[1]{
  \posttitle{
    \begin{center}\large#1\end{center}
    }
}

\setlength{\droptitle}{-2em}

  \title{I Didn't Sign Up for This: Repression and the Fragmentation of Regime Forces}
    \pretitle{\vspace{\droptitle}\centering\huge}
  \posttitle{\par}
    \author{David Bowden\\
University of Pennsylvania\\
\href{mailto:davidbow@sas.upenn.edu}{\nolinkurl{davidbow@sas.upenn.edu}}}
    \preauthor{\centering\large\emph}
  \postauthor{\par}
      \predate{\centering\large\emph}
  \postdate{\par}
    \date{November 29, 2019}

\usepackage{setspace}
\usepackage{longtable}

\usepackage{float}
\let\origtable\table
\let\endorigtable\endtable
\renewenvironment{table}[1][2] {
    \singlespacing
    \expandafter\origtable\expandafter[H]
} {
    \endorigtable
}

\raggedbottom

\begin{document}
\maketitle
\begin{abstract}
Recent high-profile examples such as the Free Syrian Army and M23 suggest that when governments violate human rights, they risk spurring resistance within their own security forces. Does repression generally lead to regime coups and rebellions originating from the regime? Under what conditions are we most likely to observe this process? I argue that when governments engage in repression, they tend to lose legitimacy at both the domestic and international levels, increasing the risk of defections from the regime. This risk should be even greater when significant numbers of soldiers share ethnic ties with the individuals being repressed, and when the military has limited centralized control over its members. Using a global sample spanning the years 1946--2013, I find robust evidence that repression is associated with an increased probability of coup attempts, and limited evidence for a link to rebel groups originating from the regime. The ethnic ties hypothesis finds more support than the military centralization prediction. The results add support to previous arguments that internal backlash provides a disincentive for governments to repress.
\end{abstract}

\doublespacing

\setlength{\parindent}{1cm}

\hypertarget{introduction}{%
\section{Introduction}\label{introduction}}

\begin{itemize}
\tightlist
\item
  Examples: FSA, M23, NLA, Franco's Nationalists, Yemen?
\item
  Importance: not a huge percentage of cases, but present in many severe civil wars, and key drivers of broader conflict trajectory. Military rebels were first rebel groups in Syria, Libya, Yemen? - very possible that these cases of unrest would not have escalated to civil war had the military remained cohesive. In the DRC, M23 has prolonged the fighting in a country that otherwise seemed headed toward peace for the first time in over a decade.
\end{itemize}

When the Arab Spring protests spread to Syria in March 2011, the regime of President Bashar al-Assad quickly responded with forceful repression, including the torture and killing of a 13-year-old boy (Macleod and Flamand \protect\hyperlink{ref-Macleod2011}{2011}). While this brutality was presumably intended to deter threats to the regime, it arguably backfired by provoking the defection of a substantial portion of the regime military including Colonel Hussein Harmoush, who expressed a feeling of complicity in the government actions, saying that ``I defected from the Syrian Arab army and took responsibility for protecting civilians\ldots{} I feel like I am responsible for the deaths of every single martyr in Syria,'' (Abouzeid \protect\hyperlink{ref-Abouzeid2011}{2011}). Riad al-Assad, who defected from the Syrian Air Force to form the Free Syrian Army rebel group, similarly declared his intent to protect protestors and resist the regime military (Lister \protect\hyperlink{ref-Lister2016}{2016}). Several other contemporary rebel groups have similar origins, including M23 in the Democratic Republic of the Congo, and the National Liberation Army in Libya, suggesting that the phenomenon could be widespread. This paper thus seeks to answer two question. First, does the use of repression place regimes at greater risk of desertion and coups? Second, under what conditions is repression most likely to produce such outcomes?

The fragmentation of the regime military is a key dynamic in many conflicts and potentially explains why the Arab Spring led to civil war in Syria and Libya, but not in other countries. Similar processes have occurred in numerous other cases, as original data presented herein shows that more than 15\% of rebel groups since World War II have traced their origins to the regime military, and an additional 9\% were founded by civilian regime officials. These rebellions have received little attention from scholars, and while some existing theories of civil war onset such as greed theory ({\textbf{???}}) could potentially account for them, many, including theories focusing on ethnic discrimination (e.g. Cederman, Wimmer, and Min \protect\hyperlink{ref-Cederman2010}{2010}) and protest escalation (e.g. Pierskalla \protect\hyperlink{ref-Pierskalla2010}{2010}), assume that rebellions originate outside the government. Studying these cases thus offers the possibility of enhancing our understanding of civil war onset.

Another prominent form of regime fragmentation --- coups d'etat --- has been the subject of much scholarship. While most of the existing literature focuses on broader structural conditions affecting coup risk, some identify a connection between protests and coup occurrence (Casper and Tyson \protect\hyperlink{ref-Casper2014}{2014}; Johnson and Thyne \protect\hyperlink{ref-Johnson2018}{2018}). Hendrix and Salehyan (\protect\hyperlink{ref-Hendrix2017}{2017}) consider the government's response to protests, finding that the possibility of regime fragmentation often deters the use of repression. Yet, fragmentation does occur, suggesting the need for further research analyzing fragmentation as a dependent variable. Additionally, deterrence effects create the possibility of endogeneity, but existing studies have not fully corrected for this concern. This study advances the literature on coups and repression by making coups a dependent variable, by accounting for the possibility of endogeneity through the use of an instrumental variable, and by examining a wider set of cases than previous studies.

This research also contributes to the literature on human rights. Understandably, most work in this area has focused on the causes of human rights violations. Several scholars, however, have turned their attention to the consequences of human rights violations for outcomes such as foreign direct investment (Blanton and Blanton \protect\hyperlink{ref-Blanton2007}{2007}) and foreign aid (Lebovic and Voeten \protect\hyperlink{ref-Lebovic2009}{2009}), and others have suggested that repression could provoke infighting amongst regime factions (Hendrix and Salehyan \protect\hyperlink{ref-Hendrix2017}{2017}). These consequences of human rights violations could offer insight to preventing abuses in the future. For example, if regime fragmentation has the potential to constrain abusive behavior (Hendrix and Salehyan \protect\hyperlink{ref-Hendrix2017}{2017}), disrupting the flow of private benefits to soldiers might undermine solidarity and strengthen this effect. By comparing specific mechanisms linking repression to regime fragmentation, this study offers the prospect of such policy recommendations.

I proceed with a review of the literature on regime fragmentation, including coups, rebellions, and desertion. Next, I articulate three theoretical processes that could link repression to regime fragmentation. I then specify a research design to test these propositions, and present results from fixed-effects and instrumental variables regression models. I conclude by situating the results in the broader literature, and by offering suggestions for future research.

\hypertarget{conceptualizing-military-defection}{%
\section{Conceptualizing Military Defection}\label{conceptualizing-military-defection}}

Few scholars have examined the phenomenon of rebel groups emerging from the regime military. However, several similar processes have been explored in the literature. Albrecht and Koehler (\protect\hyperlink{ref-Albrecht2018}{2018}) distinguish between atomized and collective

In this section I review these concepts and discuss their relation to military rebellion.

\hypertarget{rebellion}{%
\subsection{Rebellion}\label{rebellion}}

These definitions of rebellion do not preclude the possibility of such groups emerging from the state. However, the point at which an actor becomes disassociated with the state has not been fully articulated.

\hypertarget{coups-duxe9tat}{%
\subsection{Coups d'état}\label{coups-duxe9tat}}

\begin{itemize}
\tightlist
\item
  Does address dissent in regime
\item
  But doesn't account for action outside the regime.
\end{itemize}

Scholarly definitions of coups d'état often explicitly exclude revolts that occur outside existing military structures. For instance, Powell and Thyne (\protect\hyperlink{ref-Powell2011}{2011}, 252) define coups as ``illegal and overt attempts by the military or other elites \emph{within the state apparatus} to unseat the sitting executive {[}emphasis added{]}.'' As such, cases of military rebellion including the Free Syrian Army and M23 are not coded as coups in their widely-used dataset. Similarly, many theories of coups focus on the behavior of military or political elites (e.g. Roessler \protect\hyperlink{ref-Roessler2016}{2016}; Bove and Rivera \protect\hyperlink{ref-Bove2015a}{2015}; Casper and Tyson \protect\hyperlink{ref-Casper2014}{2014}). Many coups, however, are initiated by non-elite members of the military. Singh (\protect\hyperlink{ref-Singh2014}{2014}) distinguishes between coups from the top (military elites), middle (unit commanders), and bottom (enlisted men), finding that coups initiated by lower-ranking military members tend to be motivated by their treatment within the military rather than broader political concerns. While the coups from the bottom in Singh's case studies largely resemble traditional coups in the sense that the plotters attempted to seize control of the existing military apparatus, he hints at the possibility tactical diversity in this category, noting that low-ranking officers lack the ability to divert existing procedures and structures to their cause. Albrecht and Eibl (\protect\hyperlink{ref-Albrecht2018a}{2018}) similarly distinguish between coups attempted by elite and combat officers, finding that the two categories have different causes. Whereas the probability elite officer coups is largely a function of the structure of the regime and military, combat officer coups are associated with societal concerns such as low levels of welfare spending and the absence of political liberalization.

In summary, this literature shows that revolt can come from any level of the military. While these works focus on cases where this dissent manifests in attempts to seize control of the regime from within, they raise the possibility that dissident soldiers, especially those from the lower ranks, could form organizations that challenge the state from the outside. Indeed, Singh (\protect\hyperlink{ref-Singh2014}{2014}) finds that coups typically result from covert organizations of plotters within the military.

\hypertarget{mutiny}{%
\subsection{Mutiny}\label{mutiny}}

Nepstad

Book

\hypertarget{defection}{%
\subsection{Defection}\label{defection}}

Seymour

Staniland

Christia

\hypertarget{desertion}{%
\subsection{Desertion}\label{desertion}}

\hypertarget{theoretical-framework}{%
\section{Theoretical Framework}\label{theoretical-framework}}

Definition - rebellion that is launched primarily by former military members. This is the key stage in the life cycle (see Lewis 2016). Whether they recruit more broadly later is a separate question.

\hypertarget{the-sources-of-military-cohesion}{%
\subsection{The Sources of Military Cohesion}\label{the-sources-of-military-cohesion}}

I conceptualize states as having four categories of tools for producing a compliant military. First, can tailor their recruiting processes to select for soldiers who are likely to be loyal. For example, Bahrain has built a military composed almost exclusively of Sunni Muslims, while the majority Shia population is barred from service (Lutterbeck \protect\hyperlink{ref-Lutterbeck2013}{2013}; Barany \protect\hyperlink{ref-Barany2011}{2011}). This common sectarian background contributes to cohesion both among soldiers and between the military and the Sunni ruling class, and is a basis for division between soldiers and the predominantly Shia population (Bellin \protect\hyperlink{ref-Bellin2012}{2012}). Furthermore, stacking the security forces with members of minority ethnic or religious groups induces loyalty through the fear that excluded groups will engage in reprisals should the regime fall (Heger and Salehyan \protect\hyperlink{ref-Heger2007a}{2007}; McLauchlin \protect\hyperlink{ref-McLauchlin2018}{2018}). Even states with ostensibly inclusive military institutions often reject recruits perceived as having questionable loyalty. For instance, while Israeli law calls for near-universal conscription, in practice, Palestinian Arabs with Israeli citizenship are exempt (Røislien \protect\hyperlink{ref-Roislien2013}{2013}). Many states also apply screening at the individual level, using interviews or background checks to assess potential disloyalty.

Second, the promotion of loyalty is typically a core goal of military training and socialization. Militaries deliberately engineer their training programs to maximize feelings of group solidarity, patriotism, and deference to authority. In Rwanda, military recruits (Jowell)

Some training and lots of socialization tailored to one enemy

Combine 2-3 - Castillo is basically arguing that national symbols are a source of cohesion

A third, related strategy is the use of normative and symbolic power. States can employ these devices to promote loyalty, but also to deny legitimacy to potential challengers.

Lastly, material incentives play a crucial role in controlling the military. These include formal positive inducements such as salary and housing, informal positive inducements such as opportunities to engage in corruption or plunder, and negative sanctions including discharge and corporal punishment.

McLauchlin - these can vary geographically

\hypertarget{repressive-shocks-and-military-fragmentation}{%
\subsection{Repressive Shocks and Military Fragmentation}\label{repressive-shocks-and-military-fragmentation}}

\begin{itemize}
\tightlist
\item
  Militaries tend to be good at ensuring compliance for action against existing/long-standing threats.

  \begin{itemize}
  \tightlist
  \item
    Public support
  \item
    Institutionalization of procedures, culture
  \end{itemize}
\item
  Militaries tend to be less good at ensuring compliance for action against new threats, especially domestic ones.

  \begin{itemize}
  \tightlist
  \item
    Public opposition. In the abstract, repression of fellow citizens unpopular even if public would be supportive of action against actualized dissent.
  \item
    Prep for pre-existing threats limits capacity for addressing future threats.
  \end{itemize}

  Defection is a function of the regime's preparedness to repress the internal threats it faces. This preparedness in turn is a function of two factors:

  \begin{itemize}
  \tightlist
  \item
    The novelty of the threat. Governments go to great lengths to recruit and socialize their security forces in such a way that they will carry out orders to fight their enemies of the time. But if the targets of military action change over time, it may be harder for the regime to ensure compliance, particularly if the new targets are domestic civilians.
  \item
    The volume of targets. Regimes often have certain units that they view as particularly loyal, often the secret police. If the threats to the regime can be handled by these units, the risk of defection should be minimized. But if the threat is large enough that regular military forces must be deployed, the risk of defection increases dramatically.
  \end{itemize}
\end{itemize}

Sincerity. This effect could be the result of sincere concern for human rights, or the opportunistic use of bad behavior by the regime to claim legitimacy.

McLauchlin's concept of control - predictability could be another determinant. Institutionalized repression is predictable - leadership will aniticipate and prevent many opportunities for defections. The dynamics of repression shocks are much harder to predict.

While repressive shocks can create opportunities for defection by undermining the regime's compliance mechanisms, opportunity alone is insufficient for conflict (Most and Starr \protect\hyperlink{ref-Most1989}{1989}). If repression is to produce defection, it must also create willingness among soldiers to turn against the regime. I argue that this should in fact occur for a variety of reasons.

\begin{itemize}
\tightlist
\item
  Sincere humanitarian concerns

  \begin{itemize}
  \tightlist
  \item
    Soldiers may have enlisted to fight interstate foes, terrorists, existing rebels. They are socialized to dehumanize these entities. Turning their weapons on civilians is not what they signed up for. They haven't been socialized to dehumanize protesters, other civilians.
  \end{itemize}
\item
  Ethnic ties
\item
  Concerns about prosecution from domestic or international courts
\item
  Concerns about reprisals from rebels, militias, etc.
\item
  Opportunism - increased odds of obtaining international legitimacy and support
\end{itemize}

\hypertarget{other-pathways-to-military-fragmentation}{%
\subsection{Other Pathways to Military Fragmentation}\label{other-pathways-to-military-fragmentation}}

I make no claim that repression shocks are the only source of military rebellion. Other mechanisms leading to military fragmentation surely exist. Furthermore, these mechanisms are not mutually-exclusive, and may interact in interesting ways. Thus, I explore what I expect to be the most common alternative paths to fragmentation.

\begin{itemize}
\tightlist
\item
  Opportunism
\item
  Inability to employ optimal methods for cohesion

  \begin{itemize}
  \tightlist
  \item
    No minority group large enough to meet security needs
  \item
    Competing concerns/political bargains lead to conscription
  \item
    Rough/large terrain
  \end{itemize}

  \hypertarget{the-form-of-military-defection}{%
  \subsection{The Form of Military Defection}\label{the-form-of-military-defection}}

  \begin{itemize}
  \tightlist
  \item
    Security - maybe punishing deserters incentives collective defection?
  \item
    Incentives
  \item
    Coup-proofing
  \item
    Geographic location
  \end{itemize}
\end{itemize}

\hypertarget{references}{%
\section*{References}\label{references}}
\addcontentsline{toc}{section}{References}

\markboth{REFERENCES}{}

\indent

\setlength{\parindent}{-0.2in}
\setlength{\leftskip}{0.2in}
\setlength{\parskip}{8pt}

\singlespacing

\hypertarget{refs}{}
\leavevmode\hypertarget{ref-Abouzeid2011}{}%
Abouzeid, Rania. 2011. ``The Soldier Who Gave up on Assad to Protect Syria's People.'' \emph{Time}. https://content.time.com/time/world/article/0,8599,2077348,00.html.

\leavevmode\hypertarget{ref-Albrecht2018a}{}%
Albrecht, Holger, and Ferdinand Eibl. 2018. ``How to Keep Officers in the Barracks: Causes, Agents, and Types of Military Coups.'' \emph{International Studies Quarterly}, April.

\leavevmode\hypertarget{ref-Albrecht2018}{}%
Albrecht, Holger, and Kevin Koehler. 2018. ``Going on the Run: What Drives Military Desertion in Civil War?'' \emph{Security Studies} 27 (2): 179--203.

\leavevmode\hypertarget{ref-Barany2011}{}%
Barany, Zoltan. 2011. ``The Role of the Military.'' \emph{Journal of Democracy} 22 (4): 24--35.

\leavevmode\hypertarget{ref-Bellin2012}{}%
Bellin, Eva. 2012. ``Reconsidering the Robustness of Authoritarianism in the Middle East: Lessons from the Arab Spring.'' \emph{Comparative Politics} 44 (2): 127--49.

\leavevmode\hypertarget{ref-Blanton2007}{}%
Blanton, Shannon Lindsey, and Robert G. Blanton. 2007. ``What Attracts Foreign Investors? An Examination of Human Rights and Foreign Direct Investment.'' \emph{The Journal of Politics} 69 (1): 143--55.

\leavevmode\hypertarget{ref-Bove2015a}{}%
Bove, Vincenzo, and Mauricio Rivera. 2015. ``Elite Co-Optation, Repression, and Coups in Autocracies.'' \emph{International Interactions} 41 (3): 453--79.

\leavevmode\hypertarget{ref-Casper2014}{}%
Casper, Brett Allen, and Scott A. Tyson. 2014. ``Popular Protest and Elite Coordination in a Coup d'état.'' \emph{Journal of Politics} 76 (2): 548--64.

\leavevmode\hypertarget{ref-Cederman2010}{}%
Cederman, Lars-Erik, Andreas Wimmer, and Brian Min. 2010. ``Why Do Ethnic Groups Rebel?: New Data and Analysis.'' \emph{World Politics} 62 (1): 87--98.

\leavevmode\hypertarget{ref-Heger2007a}{}%
Heger, Lindsay, and Idean Salehyan. 2007. ``Ruthless Rulers: Coalition Size and the Severity of Civil Conflict.'' \emph{International Studies Quarterly} 51 (2): 385--403.

\leavevmode\hypertarget{ref-Hendrix2017}{}%
Hendrix, Cullen S., and Idean Salehyan. 2017. ``A House Divided: Threat Perception, Military Factionalism, and Repression in Africa.'' \emph{Journal of Conflict Resolution} 61 (8): 1653--81.

\leavevmode\hypertarget{ref-Johnson2018}{}%
Johnson, Jaclyn, and Clayton L. Thyne. 2018. ``Squeaky Wheels and Troop Loyalty: How Domestic Protests Influence Coups d'état, 19512005.'' \emph{Journal of Conflict Resolution} 62 (3): 597--625.

\leavevmode\hypertarget{ref-Lebovic2009}{}%
Lebovic, James H., and Erik Voeten. 2009. ``The Cost of Shame: International Organizations and Foreign Aid in the Punishing of Human Rights Violators.'' \emph{Journal of Peace Research} 46 (1): 79--97.

\leavevmode\hypertarget{ref-Lister2016}{}%
Lister, Charles. 2016. ``The Free Syrian Army: A Decentralized Insurgent Brand.'' \emph{Brookings Project on U.S. Relations with the Islamic World}, no. 26.

\leavevmode\hypertarget{ref-Lutterbeck2013}{}%
Lutterbeck, Derek. 2013. ``Arab Uprisings, Armed Forces, and CivilMilitary Relations.'' \emph{Armed Forces \& Society} 39 (1): 28--52.

\leavevmode\hypertarget{ref-Macleod2011}{}%
Macleod, Hugh, and Annasofie Flamand. 2011. ``Tortured and Killed: Hamza Al-Khateeb, Age 13.'' \emph{Al Jazeera}. https://www.aljazeera.com/indepth/features/2011/05/201153185927813389.html.

\leavevmode\hypertarget{ref-McLauchlin2018}{}%
McLauchlin, Theodore. 2018. ``The Loyalty Trap: Regime Ethnic Exclusion, Commitment Problems, and Civil War Duration in Syria and Beyond.'' \emph{Security Studies} 27 (2): 296--317.

\leavevmode\hypertarget{ref-Most1989}{}%
Most, Benjamin A, and Harvey Starr. 1989. \emph{Inquiry, Logic and International Politics}. Columbia, SC: University of South Carolina Press.

\leavevmode\hypertarget{ref-Pierskalla2010}{}%
Pierskalla, Jan Henryk. 2010. ``Protest, Deterrence, and Escalation: The Strategic Calculus of Government Repression.'' \emph{Journal of Conflict Resolution} 54 (1): 117--45.

\leavevmode\hypertarget{ref-Powell2011}{}%
Powell, Jonathan M., and Clayton L. Thyne. 2011. ``Global Instances of Coups from 1950 to 2010: A New Dataset.'' \emph{Journal of Peace Research} 48 (2): 249--59.

\leavevmode\hypertarget{ref-Roessler2016}{}%
Roessler, Philip. 2016. \emph{Ethnic Politics and State Power in Africa: The Logic of the Coup-Civil War Trap}. Cambridge, United Kingdom ; New York, NY: Cambridge University Press.

\leavevmode\hypertarget{ref-Roislien2013}{}%
Røislien, Hanne Eggen. 2013. ``Religion and Military Conscription: The Case of the Israel Defense Forces (IDF).'' \emph{Armed Forces \& Society} 39 (2): 213--32.

\leavevmode\hypertarget{ref-Singh2014}{}%
Singh, Naunihal. 2014. \emph{Seizing Power}. Johns Hopkins University Press.


\end{document}
